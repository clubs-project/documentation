\documentclass[a4paper,11pt]{article}

\usepackage[utf8]{inputenc}
\usepackage{booktabs, array, pdflscape}
\usepackage{geometry}
\usepackage{graphics,subfigure,graphicx}
\usepackage{color}
\usepackage{url}
\usepackage{enumerate}

\setlength{\textheight}{24cm}  
\setlength{\textwidth}{15cm}
\setlength\oddsidemargin{0cm}
\setlength\evensidemargin{0cm}
\setlength\voffset{-1cm}

\renewcommand{\textfraction}{0.01}
\renewcommand{\floatpagefraction}{0.75}
\renewcommand{\topfraction}{0.8}
\renewcommand{\bottomfraction}{0.8}

\newcommand{\red}[1]{\textcolor{red}{#1}}	
\newcommand{\mc}[3]{\multicolumn{#1}{#2}{#3}}

\newcommand{\Ni}{({\em i\,})~}
\newcommand{\Nii}{({\em ii\,})~}
\newcommand{\Niii}{({\em iii\,})~}

%opening
\title{
\includegraphics[width=3cm]{./img/200px-SuitClubs.png} \\
\Huge Query Translation and \\ Integration Approaches \\ 
}
\author{\vspace*{1cm}\\ \LARGE Cristina Espa\~na-Bonet \medskip \\ \Large Universit\"at des Saarlandes}
\date{\vspace*{2cm} -- v1.0 --\\January 2018}


\begin{document}

\clearpage\maketitle
\thispagestyle{empty}

\vspace*{5cm}
\begin{abstract}
This document describes 
\end{abstract}

\newpage
\tableofcontents
\clearpage

% guarrada, no va el \cleardoublepage
% \clearpage\mbox{}\clearpage

%\newpage
% \section{Introduction}
% \label{s:intro}

\section{Controlled Terms}
\label{s:ct}
% \section{MeSH Multilingual Lexicon}
% \label{s:mesh}

We extract all the controlled terms in the frozen Solr instance "pubpsych-core"%
\footnote{Database as in 4th August 2017.}  using the fields {\tt CTDL}, {\tt CTEL}, {\tt CTFL} and {\tt CTSL} \red{is this correct?}. Table~\ref{tab:ct} shows the statistics per language. Notice that there are no {\tt CTSL} in Spanish but we did not retrieve any result for {\tt CTSH} or {\tt CTS} either.

Some of the entries have two parts, the descriptor and a class specification in parentheses:
{\small 
\begin{verbatim}
   Action Potentials
   Action Potentials (drug effects)
   Action Potentials (genetics)
\end{verbatim}
}

This allows to further split the controlled terms and diminish the number of unique terms to translate as seen in rows \emph{uniq no()} and \emph{uniq only()} of Table~\ref{tab:ct}.


\begin{table}[t]
\centering
\begin{tabular}{lrrrr}
  \toprule
         & \mc{1}{c}{German} & \mc{1}{c}{English} & \mc{1}{c}{French} & \mc{1}{c}{Spanish}\\
  \midrule
    CT$lan$L           & 3,659,210 & 4,639,171 & 2,371,110 & 0\\
    CT$lan$L uniq      &    56,754 &    60,939 &    51,759 & 0\\
    CT$lan$L uniq no() &    23,556 &    27,734 &    18,623 & 0\\
    CT$lan$L uniq only() &     393 &       392 &       187 & 0\\
  \midrule
     MeSH               & 70,694 & 175,004 & 96,333 & 66,828\\
     WP (health+phsy.)  & 81,369 & 80,762  & 80,285 & 81,059\\
  \bottomrule
 \end{tabular}
\caption{Number of controlled terms per language in the PubPsych Database (top rows). Number of aligned controlled terms per language in our multilingual resources (bottom rows).}
\label{tab:ct}
\end{table} 

\subsection{Coverage by MeSH (and in-domain WP titles)}
\label{ss:mesh}

In order to see the coverage of the controlled terms by our resources we first extract a list per language of the terms that represent a concept in the multilingual MeSH. For all these terms we have the 4-language-translation. See the number of elements in the MeSH row in Table~\ref{tab:ct}.

We lowercase all the entries and look for the unique elements in our CTs that cannot be translated by the MeSH thesauri (MeSH $unk$ row in Table~\ref{tab:coverage}). We do the same with a list of aligned in-domain Wikipedia titles (WP $unk$) and with the combination of both resources (MeSH+WP $unk$). The translatable elements cover almost all the French CTs (98\%) but only a 78\% and  73\% for German and English ones respectively. The class that appears in parentheses must be also translated independently \red{(TODO grab statistics)}

\begin{table}[t]
\centering
\begin{tabular}{lrrrr}
  \toprule
         & \mc{1}{c}{German} & \mc{1}{c}{English} & \mc{1}{c}{French} & \mc{1}{c}{Spanish}\\
  \midrule
   \midrule
    CT$lan$L uniq no()    & 23,556 & 27,734 & 18,623 & 0\\
    ~~-MeSH $unk$         &  4,704 &  7,387 &    221 & 0 \\
    ~~-WP $unk$           & 15,564 & 18,421 & 11,096 & 0\\
    ~~-MeSH+WP $unk$      &  4,271 &  6,724 &    197 & 0 \\
   \midrule
    CT$lan$L no()         & 3,659,210 &  4,639,171 & 2,371,110  & 0\\
    ~~-MeSH+WP $covered$  & 2,861,446 &  3,366,307 & 2,322,733  & 0 \\
                          &    (78\%) &     (73\%) &     (98\%) & 0 \\

    
  \bottomrule
 \end{tabular}
\caption{Untranslatable terms of \emph{CT$lan$L no()} by MeSH and WP thesauri after lowercasing the entries.}
\label{tab:coverage}
\end{table} 



\section{Quadrilingual Lexicon}
\label{s:lexicon}

\subsection{Multilingual MeSH}
\label{ss:meshLex}

We extract the quadrilingual lexicon from {\tt MeSH\_2017\_de+en+fr+es.xml} in an appropriate format for query translation. We extract one list per language L1, where for each term (preferred, non-preferred, and permutations) describing a concept in L1 only the preferred term in the other languages L2, L3 and L4 is added as translation. The identifier of the concept is also added.

Example for the English terms for concept ID:M0000003: 
{\small 
\begin{verbatim}
   Slaughter Houses|||es:Mataderos|||de:Schlachthöfe|||fr:Abattoirs|||ID:M0000003
   House, Slaughter|||es:Mataderos|||de:Schlachthöfe|||fr:Abattoirs|||ID:M0000003
   Houses, Slaughter|||es:Mataderos|||de:Schlachthöfe|||fr:Abattoirs|||ID:M0000003
   Slaughter House|||es:Mataderos|||de:Schlachthöfe|||fr:Abattoirs|||ID:M0000003
   Slaughterhouses|||es:Mataderos|||de:Schlachthöfe|||fr:Abattoirs|||ID:M0000003
\end{verbatim}
}

Within a language, the elements are unique but there might be degeneracy when we concatenate the 4 languages.

\subsection{Multilingual Wikipedia Titles}
\label{ss:wpLex}

\red{TODO: text in year 1 pdf.}

\red{TODO: conver union file to MeSH format}

\section{}
\label{s:}



% \section{Conclusions}
% \label{s:conclusions}


%
% ---- Bibliography ----
%
\addcontentsline{toc}{section}{References}
\bibliographystyle{plain}
\bibliography{genericMT}


\end{document}
