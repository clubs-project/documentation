\documentclass[a4paper,11pt]{article}

\usepackage[utf8]{inputenc}
\usepackage{booktabs, array, pdflscape}
\usepackage{geometry, rotating}
\usepackage{graphics,subfigure,graphicx}
\usepackage{color}
\usepackage{url}
\usepackage{enumerate}
\usepackage{enumitem}
\usepackage{multirow}
\usepackage{amssymb}
\usepackage{listings}
\usepackage[multiple]{footmisc}
\usepackage{tikz}
\usetikzlibrary{shapes}

\setlength{\textheight}{24cm}  
\setlength{\textwidth}{15cm}
\setlength\oddsidemargin{0cm}
\setlength\evensidemargin{0cm}
\setlength\voffset{-1cm}

\renewcommand{\textfraction}{0.01}
\renewcommand{\floatpagefraction}{0.75}
\renewcommand{\topfraction}{0.8}
\renewcommand{\bottomfraction}{0.8}

\newcommand{\red}[1]{\textcolor{red}{#1}}	
\newcommand{\mc}[3]{\multicolumn{#1}{#2}{#3}}

\newcommand{\Ni}{({\em i\,})~}
\newcommand{\Nii}{({\em ii\,})~}
\newcommand{\Niii}{({\em iii\,})~}

%opening
\title{
	\includegraphics[width=3cm]{./img/200px-SuitClubs.png} \\
	\Huge M3.1 -- Cross-lingual Thesaurus and \\ Controlled Term Translation \\ 
}
\author{\vspace*{1cm}\\ \LARGE Cristina Espa\~na-Bonet$^{1}$, Roland Ramthun$^{2}$ and Sophie Henning$^{1}$ \medskip \\ 
	\Large $^{1}$Universit\"at des Saarlandes\\ \Large $^{2}$Leibniz-Zentrum für Psychologische Information und Dokumentation}
\date{\vspace*{2cm} -- v1.2 --\\March 2018}


\begin{document}
	
	\clearpage\maketitle
	\thispagestyle{empty}
	
	\vspace*{5cm}
	\begin{abstract}
		This document describes the data, resources, methodology and software developed to translate the controlled terms and related text available as metadata in the PubPsych database.
	\end{abstract}
	
	\newpage
	\tableofcontents
	\clearpage
	
	% guarrada, no va el \cleardoublepage
	% \clearpage\mbox{}\clearpage
	
	%\newpage
	% \section{Introduction}
	% \label{s:intro}
	
	\section{Controlled Terms in PubPsych}
	\label{s:ct}
	% \section{MeSH Multilingual Lexicon}
	% \label{s:mesh}
	
	The different database segments in PubPsych use controlled terms from different systems (e.g. the Medical Subject Headings or the APA thesaurus) or even no controlled terminology at all. Many databases include indexing terms, meaning these terms are descriptive, but freely assigned and not from a controlled vocabulary.
	
	\bigskip
	\noindent The relevant fields holding the controlled term and indexing term information are
		\begin{description}
		\item[CT\mbox{\boldmath $lan$}H:] "Controlled term high". These are terms from controlled vocabulary (MeSH, APA/PSYNDEX terms, etc.), not freely assigned terms.
		\item[CT\mbox{\boldmath $lan$}L:] "Controlled terms low". As CT$lan$H, but the person who created the record gave these entries a lower importance for describing the content than the ones in CT$lan$H.
		\item[IT\mbox{\boldmath $lan$}H:] "Additional descriptor high". As the name says, additional describing terms, which may have been freely chosen by the person who created the record, so they do not need to come from a controlled terminology.
		\item[IT\mbox{\boldmath $lan$}L:] "Additional descriptor low". As IT$lan$H, but with lower descriptive relevance.
	\end{description}
	
	Table~\ref{tab:cts-per-db} and Table~\ref{tab:its-per-db} give an overview about the available descriptive term data available for each database segment.
	
		\begin{table}[h]
		\centering
		\begin{tabular}{lrrrr}
			\toprule
			& \mc{1}{c}{CTEH/CTEL} & \mc{1}{c}{CTDH/CTDL} & \mc{1}{c}{CTFH/CTFL} & \mc{1}{c}{CTSH/CTSL}\\
			\midrule
			MEDLINE           & 90.4\%/96.0\% & 90.4\%/96.0\% & 90.4\%/96.0\% & -/-\\
			ERIC              & 53.2\%/98.9\% &    -/- &    -/- & -/-\\
			ISOC   &    -/- &    -/- &    -/- & -/-\\
			NARCIS &       -/- &       -/- &       -/- & -/-\\
			NORART &       -/94.2\% &       -/- &       -/- & -/-\\
			PASCAL &       -/- &       -/- &       -/- & -/-\\
			PsychData &       -/94.3\% &       -/94.3\% &       -/- & -/-\\
			PsychOpen &       -/- &       -/- &       -/- & -/-\\
			PSYNDEX &       96.0\%/93.2\% &       96.0\%/93.2\% &       -/- & -/-\\
			\bottomrule
		\end{tabular}
		\caption{Relative availability of controlled terms for records per database segment}
		\label{tab:cts-per-db}
	\end{table}

		\begin{table}[h]
	\centering
	\begin{tabular}{lrrrr}
		\toprule
		& \mc{1}{c}{ITEH/ITEL} & \mc{1}{c}{ITDH/ITDL} & \mc{1}{c}{ITFH/ITFL} & \mc{1}{c}{ITSH/ITSL}\\
		\midrule
		MEDLINE    & 9.9\%/8.9\% & -/- & -/- & -/-\\
		ERIC       &    43.6\%/20.9\% &    -/- &    -/- & -/-\\
		ISOC   &    -/- &    -/- &    -/- & -/95.9\%\\
		NARCIS &       40.8\%/- &       -/- &       -/- & -/-\\
		NORART &       -/34.6\% &       -/- &       -/- & -/-\\
		PASCAL &       -/99.4\% &       -/0\% &       -/99.4\% & -/99.2\%\\
		PsychData &       -/- &       -/- &       -/- & -/-\\
		PsychOpen &       80.8\%/- &       -/- &       -/- & -/-\\
		PSYNDEX &       4.4\%/1.3\% &       4.5\%/1.3\% &       -/- & -/-\\
		\bottomrule
	\end{tabular}
	\caption{Relative availability of indexing terms for records per database segment}
	\label{tab:its-per-db}
\end{table}

	For a first analysis, we extract some of the controlled terms (CTs) from a frozen Solr instance%
	\footnote{PubPsych record set as of 4th August 2017 with 1,037,536  entries.}  using the fields {\tt CTDL}, {\tt CTEL}, {\tt CTFL} and {\tt CTSL}. Table~\ref{tab:ct} shows the statistics per language. We use CT$lan$L for this analysis because it is the field appearing in more records.
% 	Notice that there are no {\tt CTSL} (Spanish CTs) but we did not retrieve any result for {\tt CTSH} either, the other field with information related to controlled terms.
	
	Some of the entries have two parts, the descriptor and a class specification in parentheses:
	{\small 
		\begin{verbatim}
		Action Potentials
		Action Potentials (drug effects)
		Action Potentials (genetics)
		\end{verbatim}
	}
	
	This allows to further split the controlled terms into a descriptor and a specification thereby reducing the number of unique terms to translate as seen in rows \emph{uniq descriptors} and \emph{uniq specifications} of Table~\ref{tab:ct}.
	
	
	\begin{table}[h]
		\centering
		\begin{tabular}{lrrrr}
			\toprule
			& \mc{1}{c}{German} & \mc{1}{c}{English} & \mc{1}{c}{French} & \mc{1}{c}{Spanish}\\
			\midrule
			CT$lan$L total           & 3,659,210 & 4,639,171 & 2,371,110 & 0\\
			CT$lan$L uniq       &    56,754 &    60,939 &    51,759 & 0\\
			descriptors uniq   &    23,556 &    27,734 &    18,623 & 0\\
			specifications uniq &       393 &       392 &       187 & 0\\
			\bottomrule
		\end{tabular}
		\caption{Number of controlled terms per language in the PubPsych Database. See text for the nomenclature. CT$lan$L denotes the name of the language dependent PubPsych fields for CTs, e.g. CTDL for German or CTSL for Spanish}
		\label{tab:ct}
	\end{table} 
	
	After this preliminary analysis to study the expectable quantity of different terms, we fixed the set of  relevant fields to be CT$lan$H, CT$lan$L, IT$lan$H and IT$lan$L. In order to translate these 16 fields (4 fields per language) we create a quadrilingual lexicon as explained in the next section.
	% 
	% \subsection{Coverage by MeSH (and in-domain WP titles)}
	% \label{ss:mesh}
	% 
	% In order to see the coverage of the controlled terms by our resources we first extract a list per language of the terms that represent a concept in the multilingual MeSH. For all these terms we have the 4-language-translation. See the number of elements in the MeSH row in Table~\ref{tab:ct}.
	% 
	% We lowercase all the entries and look for the unique elements in our CTs that cannot be translated by the MeSH thesauri (MeSH $unk$ row in Table~\ref{tab:coverage}). We do the same with a list of aligned in-domain Wikipedia titles (WP $unk$) and with the combination of both resources (MeSH+WP $unk$). The translatable elements cover almost all the French CTs (98\%) but only a 78\% and  73\% for German and English ones respectively. The class that appears in parentheses must be also translated independently \red{(TODO grab statistics)}
	% 
	% \begin{table}[t]
	% \centering
	% \begin{tabular}{lrrrr}
	%   \toprule
	%          & \mc{1}{c}{German} & \mc{1}{c}{English} & \mc{1}{c}{French} & \mc{1}{c}{Spanish}\\
	%   \midrule
	%    \midrule
	%     CT$lan$L uniq no()    & 23,556 & 27,734 & 18,623 & 0\\
	%     ~~-MeSH $unk$         &  4,704 &  7,387 &    221 & 0 \\
	%     ~~-WP $unk$           & 15,564 & 18,421 & 11,096 & 0\\
	%     ~~-MeSH+WP $unk$      &  4,271 &  6,724 &    197 & 0 \\
	%    \midrule
	%     CT$lan$L no()         & 3,659,210 &  4,639,171 & 2,371,110  & 0\\
	%     ~~-MeSH+WP $covered$  & 2,861,446 &  3,366,307 & 2,322,733  & 0 \\
	%                           &    (78\%) &     (73\%) &     (98\%) & 0 \\
	%   \bottomrule
	%  \end{tabular}
	% \caption{Untranslatable terms of \emph{CT$lan$L no()} by MeSH and WP thesauri after lowercasing the entries.}
	% \label{tab:coverage}
	% \end{table} 
	% 
	
	
	\section{Quadrilingual Lexicon}
	\label{s:lexicon}
	
	The resources described in this section can be found in the project's Seafile in the folder:
	{\tt CLIR-PubPsych/Code/MT/DBtranslator/models/CT} 
	
	%TODO No use of multilingual APA thesaurus
	
	\subsection{Multilingual MeSH}
	\label{ss:meshLex}
	
	We extract the largest part of our quadrilingual lexicon from a quadrilingual MeSH version created with MeSHMerger\footnote{\url{https://github.com/clubs-project/MeSHMerger}} {\tt file MeSH\_2017\_de+en+fr+es.xml}. The format of the data has been changed to a list format for CT translation. We extract one list per language L1, where for each term (preferred, non-preferred, and permutations) describing a concept in L1 only the preferred term in the other languages L2, L3 and L4 is added as translation. This ensures that any term for any concept in any language is always mapped to the preferred term in the other languages. The identifier of the concept is also added. With this procedure we obtain 175,004 concepts for English, 96,333 for French, 70,694 for German and 66,828 for Spanish. The difference between languages stems from the different number of \emph{synonyms} (permutations and strings) in the MeSH translations.
	
	\bigskip
	\noindent Example for the English terms for concept {\tt ID:M0000020}. We first show the complete MeSH entry for the concept, and then the four files that are generated were one can see why different languages have different numbers of entries:
	
	{\small 
	\lstset{language=XML}
	\begin{lstlisting} 
   MeSH_2017_de+en+fr+es.xml:
	<concept id="M0000020">
	<term id="T000045" lang="eng" preferred="true">
	<string>Abomasum</string>
	<permutation>Abomasums</permutation>
	</term>
	<term id="spa0000603" lang="spa" preferred="true">
	<string>Abomaso</string>	
	</term>
	<term id="spa0049997" lang="spa" preferred="false">
	<string>Cuajar</string>
	</term>
	<term id="ger0000018" lang="ger" preferred="true">
	<string>Labmagen</string>
	</term>
	<term id="fre0063293" lang="fre" preferred="true">
	<string>Abomasum</string>
	</term>
	<term id="fre0000018" lang="fre" preferred="false">
	<string>Caillette</string>
	</term>
	</concept>
	\end{lstlisting}
		
	\begin{lstlisting} 
   mesh.dekey.txt:
	Labmagen|||en:Abomasum|||es:Abomaso|||fr:Abomasum|||ID:M0000020
	
   mesh.enkey.txt:
	Abomasum|||es:Abomaso|||de:Labmagen|||fr:Abomasum|||ID:M0000020
	Abomasums|||es:Abomaso|||de:Labmagen|||fr:Abomasum|||ID:M0000020
		
   mesh.eskey.txt:       
	Abomaso|||en:Abomasum|||de:Labmagen|||fr:Abomasum|||ID:M0000020
	Cuajar|||en:Abomasum|||de:Labmagen|||fr:Abomasum|||ID:M0000020
		
   mesh.frkey.txt:  
	Abomasum|||en:Abomasum|||es:Abomaso|||de:Labmagen|||ID:M0000020
	Caillette|||en:Abomasum|||es:Abomaso|||de:Labmagen|||ID:M0000020
	\end{lstlisting}
	}
	
	%    Slaughter Houses|||es:Mataderos|||de:Schlachthöfe|||fr:Abattoirs|||ID:M0000003
	%    House, Slaughter|||es:Mataderos|||de:Schlachthöfe|||fr:Abattoirs|||ID:M0000003
	%    Houses, Slaughter|||es:Mataderos|||de:Schlachthöfe|||fr:Abattoirs|||ID:M0000003
	%    Slaughter House|||es:Mataderos|||de:Schlachthöfe|||fr:Abattoirs|||ID:M0000003
	%    Slaughterhouses|||es:Mataderos|||de:Schlachthöfe|||fr:Abattoirs|||ID:M0000003
	
	Notice that within each file/language, the keys are unique but there might be degeneracy when we concatenate the 4 languages into a single file -- in this example, \emph{Abomasum} is a key both for English and French.
	
	\subsection{Multilingual Wikipedia Entries}
	\label{ss:wpLex}
	
	% \subsubsection{Multilingual Wikipedia In-domain Titles}
	
% 	To increase the amount of psychological term translations, we have extracted multilingual in-domain titles from Wikipedia with the WikiTailor tool\footnote{\url{https://github.com/cristinae/WikiTailor}}
% 	\cite{barronEtAl:2015}. 
% 	Lexicons have been built from aligned articles in the psychology and health domains for English, German, French and Spanish using WikiTailor models WT0.5-100 or WT0.5-500 depending on the language.
% 	%TODO Cristina, could you elaborate on which model was used for which language any why? This is hard to understand if you do not know WikiTailor

		
	To increase the amount of psychological term translations, we have extracted multilingual in-domain titles from Wikipedia on psychology and health with the WikiTailor tool\footnote{\url{https://github.com/cristinae/WikiTailor}} \cite{barronEtAl:2015}. 
	
	WikiTailor extracts domain articles by exploring the categories graph starting from the category describing the domain (psychology and health in our case) and identifying a subset of related categories and their associated articles\footnote{We use models WT0.5-100 or WT0.5-500 depending on the language. Refer to WikiTailor manual if you want to replicate these models \url{http://cristinae.github.io/WikiTailor/}}. These articles are gathered independently for English, German, French and Spanish and, afterwards, the intersection or union of the articles is done. For the intersection, we use the articles that have been identified simultaneously in the four languages. For the union, we expand the set of articles to include all the articles that have been identified as in-domain articles at least in one of the languages with the equivalent article in the other three languages in case it exists.	
	Using the intersection of in-domain articles in the four languages we obtain a high precision/low recall multilingual lexicon with 497 entries. With the union of in-domain articles we gather a low precision/high recall multilingual lexicon with 81,369 entries. 
	%TODO Which data was used, phrases from article text or titles, as said above? I think I get what you did for the high precision/low recall multilingual lexicon, but what did you do for the low precision/high recall version?
	
	% After removing duplicates, we obtain the figures of Table~\ref{tab:4lex} depending on the language.
	The lexicon contains both single words and phrases related to our domain, but in lots of cases entries correspond to named entities:
	
	\bigskip
	\begin{small}
		% \begin{tabular}[h]{rl rl rl rl}
		%   ID En    & Title En       & ID Es   & Title Es        &  ID Fr    & Title Fr            &  ID De   & Title De \\
		%   1053998  & Perception     &176025   & Percepci\'on    & 360640    &Perception           & 387646   & Wahrnehmung \\
		%   10269587 & Echoic\_memory & 4636204 & Memoria\_ecoica & 3858294   & M\'emoire\_auditive & 4334978  & Echoisches\_Ged\"achtnis \\
		% \end{tabular}
		\begin{tabular}[h]{llll}
			\toprule
			En     &Es          & Fr             & De \\
			\midrule
			Perception     & Percepci\'on      &Perception            & Wahrnehmung \\
			Echoic\_memory & Memoria\_ecoica   & M\'emoire\_auditive  & Echoisches\_Ged\"achtnis \\
			Emil\_Kraepelin & Emil\_Kraepelin  & Emil\_Kraepelin      & Emil\_Kraepelin \\
			... & ... & ... & ... \\
			\bottomrule
		\end{tabular}
	\end{small}
	
	\bigskip
	In a similar way, we extract aligned category names from Wikipedia, but this time, we select all of them and not only those related to psychology. 38,038 entries are obtained in this case.
	
	As for the MeSH lexicon, we build 4 files, one per language, with the entries in the four languages aligned. In this case though, there is no associated ID:
	
	{\small 
		\begin{verbatim}
		wp.dekey.txt:
		Wahrnehmung|||en:Perception|||es:Percepción|||fr:Perception
		Echoisches Gedächtnis|||en:Echoic memory|||es:Memoria ecoica|||fr:Mémoire auditive
		
		wp.enkey.txt:
		Perception|||es:Percepción|||de:Wahrnehmung|||fr:Perception
		Echoic memory|||es:Memoria ecoica|||de:Echoisches Gedächtnis|||fr:Mémoire auditive
		
		wp.eskey.txt:
		Percepción|||en:Perception|||de:Wahrnehmung|||fr:Perception
		Memoria ecoica|||en:Echoic memory|||de:Echoisches Gedächtnis|||fr:Mémoire auditive
		
		wp.frkey.txt:
		Perception|||en:Perception|||es:Percepción|||de:Wahrnehmung
		Mémoire auditive|||en:Echoic memory|||es:Memoria ecoica|||de:Echoisches Gedächtnis
		\end{verbatim}
	}
	
	
	\subsection{Apertium Dictionaries}
	\label{ss:apertium}
	
	Apertium~\cite{forcadaEtal:2011} is a free/open-source ruled-based translation engine that uses bilingual dictionaries for lexical transfer. We have used three of their dictionaries\footnote{\url{http://wiki.apertium.org/wiki/List_of_dictionaries}} ($en$-$de$, $en$-$es$ and $es$-$fr$) to extract a quadrilingual dictionary with the overlapping entries. Table~\ref{tab:4lex} shows the number of entries of this multilingual dictionary in comparison with the other sources.
	
	\subsection{Post-edited Automatic Translations}
	\label{ss:manual}
	
	Finally, we have selected a set of tokens within our controlled terms not covered by the previous resources and translated them with the automatic translation engine DeepL\footnote{\url{https://www.deepl.com}}  (translator as of 25th January and 1st-2nd February 2018). These $\sim$4,000 entries have been manually post-edited  mainly to improve mistranslations due to ambiguous options, but the post-editor was neither native in the four languages nor in-domain expert. Table~\ref{tab:4lex} shows the exact number of entries depending on the source language in the row "Manual". Mismatches in the numbers of one row hint to the availability of synonyms for a language.

	\subsection{WikiData}
	\label{ss:wikidata}
	
	For the final version of the lexicon we add multilingual entries from Wikidata%
	\footnote{\url{https://www.wikidata.org}}.
	
	BLABLABLA
	
	
	\begin{table}[t]
		\centering
		\begin{tabular}{lrrrr}
			\toprule
			& \mc{1}{c}{German} & \mc{1}{c}{English} & \mc{1}{c}{French} & \mc{1}{c}{Spanish}\\
			\midrule
			MeSH                     & 70,694 & 175,004 & 96,333 & 66,828\\
			%      WP (health+phsy.)  & 81,369 & 80,762  & 80,285 & 81,059\\
			WPtitles (health+phsy.)  & 81,369 & 81,369  & 81,369 & 81,369\\
			WPcategories             & 38,038 & 38,038  & 38,038 & 38,038\\
			Apertium                 &  7,792 &  5,935  &  6,020 &  5,846\\
			Manual                   &  4,262 &  4,142  &  4,047 &  4,081\\
			Wikidata (Propotype2 only)  & 5,576,686  & 5,576,686 &  5,576,686 & 5,576,686 \\
			\midrule
			\emph{Total Propotype1}  & 202,128&304,277 & 225,607 & 195,937\\
			\emph{Total Propotype2}  & 5,778,814 & 5,880,963 & 5,802,293 & 5,772,623
\\
			\bottomrule
		\end{tabular}
		\caption{Number of aligned terms per language in our multilingual resources. The row with the total excludes duplicate entries between the sources.}
		\label{tab:4lex}
	\end{table} 
	
	
	\subsection{Cleaning and Quad-lexicon Compilation}
	\label{ss:cleaning}
	We have cleaned and compiled two quadrilingual lexicons: one that only consists of entries of the MeSH dictionary and one that consists of the entries of all the other dictionaries. We have separated the sources since we consider the MeSH entries to be of higher quality.
	
	We have applied the following cleaning steps to both dictionaries:
	\begin{itemize}
		\item Lowercase tokens
		\item Remove diacritics (e.g \textit{r\"ucklauf} $\rightarrow$ \textit{rucklauf})
		\item Replace \textit{\ss} with \textit{ss} (that is how Solr deals with this character)
		\item Delete \textit{[dokumenttyp]} annotation (e.g. \textit{biografie [dokumenttyp]} $\rightarrow$ \textit{biografie})
		\item Remove the whole entry if the source word or a translation is empty
		\item Remove the whole entry if the source word or a translation is a stopword in any of the languages
	\end{itemize}
	
	Moreover, we eliminated source words that were source words in more than one language, sticking to the following order: English was favoured over German, German over French and French over Spanish. In order to cover different spellings, we have also introduced some duplicates: Source words containing umlauts were duplicated with a version in which the umlaut was replaced by the basic character and an \textit{e} (\textit{rücklauf} was not only changed to \textit{rucklauf}, but also lead to another entry with \textit{ruecklauf} as a source word). Similarly, we have added duplicates for words ending with \textit{-ise} respectively \textit{-ize}, \mbox{\textit{-isation}} respectively \textit{-ization} and \textit{-our} and \textit{-or} to account for differences between American and British English. 	Furthermore, we manually deleted some wrong entries. 
	
	After applying this procedure to each other dictionary from the sections \ref{ss:wpLex}, \ref{ss:apertium} and \ref{ss:manual}, we merged them into one dictionary, while, in the case of duplicates, following this priority setting: The lexicon built with WikiTailor was favoured over the dictionary made of Wikipedia category name alignments, that one over the post-edited automatic translations and those over the Apertium dictionary.
		
	\section{Controlled Term Translation}
	\label{s:cttrad}
	
	\subsection{Methodology}
	We use the resources described in the previous section to translate the controlled terms appearing in the articles of the PubPsych database (Section~\ref{s:ct}).
	Notice that the most accurate translation would be achieved with the multilingual MeSH alone. The other three resources add noise to the translations but significantly increase the coverage of the engine.
	
	\bigskip
	\noindent
	We follow the strategy below and sketched in Figure~\ref{fig:diagram}:
	\begin{enumerate}
		\item A CT is splitted into the descriptor and the class specification (Section~\ref{s:ct}). Both parts are subsequently cleaned and translated independently. 
		Ex: \emph{Action Potentials (genetics)} $\Rightarrow$ \emph{Action Potentials, genetics}
		\item Part Translation
		\begin{enumerate}[label*=\arabic*.]
			\item  All possible capitalisations of the part (\emph{Action Potentials}, \emph{action potentials}, \emph{Action potentials}) are looked up in the corresponding quadrilingual lexicon and, in case the entry exists, the translations into the other three languages are obtained. \\ 
			Ex: \emph{Action Potentials$|||$es:Potenciales de Acción$|||$de:Aktionspotentiale$|||$fr:Potentiels d'action}
			\item The original capitalisation is restored.
		\end{enumerate}
		
		\item Token Translation. If a part is not found in the dictionary, it is translated on a token-by-token basis.
		\begin{enumerate}[label*=\arabic*.]
			\item The part is split into tokens and words are translated independently.  
			\item  All possible capitalisations of the token are looked up in the corresponding quadrilingual lexicon and, in case the entry exists, the translations into the other three languages are obtained.
			\item If the entry is not available, some basic rules regarding the formation of plural nouns (see Appendix~\ref{ap:plural}) are applied to obtain a singular form for the entry. In case the entry exists, the translations into the other three languages for the singular form are obtained and used to translate it.
			\item If the entry is not available, we copy the source token as translation for the three other languages.
			\item The original capitalisation is restored.
		\end{enumerate}
		
		\item Tokens and parts are joined with the appropriate punctuation to build the final translation of the original CT.
		
	\end{enumerate}
	
	
   \tikzstyle{io} = [rectangle, draw=black, fill=white, rounded corners, text width=12em, text badly centered, inner sep=8pt, node distance=2.5cm]
  \tikzstyle{decision} = [diamond, draw, fill=white, text width=5em, text badly centered, node distance=3.5cm, inner sep=0pt]
  \tikzstyle{block} = [rectangle, draw=black, fill=white, node distance=2.5cm, text width=12em, text badly centered, minimum height=2.5em, inner sep=0pt]
  \tikzstyle{line} = [draw,->,-latex,black,line width=1pt]
  \begin{figure}[t!]
    \centering
    \begin{tikzpicture}[auto,draw=black,line width=1pt,scale=0.65,transform shape]
	% Place nodes
    
	\node [io] (input) {\large INPUT: Term};
	\node [block, below of=input, text width=26em, minimum height=3.5em, node distance=2cm] (init) {Generation of term alternatives\\ (\emph{Action Potentials}, \emph{action potentials}, \emph{Action potentials}) };

	\node [decision, below of=init, node distance=2.5cm] (q) {in quad-lexicon?};
	\node [block, left of=q, node distance=6cm, text width=22em, minimum height=4.5em] (extract)  {Extract entry\\(\emph{Action Potentials$|||$es:Potenciales de Acci\'on$|||$ de:Aktionspotentiale$|||$fr:Potentiels d'action})};
	\node [block, below of=extract, node distance=2.35cm] (restore1) {Restore capitalisation};
	\node [io, below of=restore1, node distance=1.9cm] (end) {\large OUTPUT: Quad-term};
	
	\node [block, below of=q, node distance=2.4cm] (split) {Split by tokens};
	\node [block, below of=split, node distance=1.8cm] (token) {Token alternatives};
	\node [decision, below of=token, node distance=2.5cm] (q2) {in quad-lexicon?};
	\node [block, right of=q2, node distance=5cm, text width=8em] (extract2)  {Extract entry};
	\node [block, below of=extract2, node distance=2.3cm] (restore2) {Restore capitalisation};
	\node [decision, below of=restore2, node distance=2.6cm] (q3) {more tokens?};
	\node [block, below of=q3, node distance=2.2cm] (restore3) {Recompose entry};
	\node [io, below of=restore3, node distance=1.9cm] (end2) {\large OUTPUT: Quad-term};
	\node [block, below of=q2, node distance=2.3cm, text width=8em] (singular)  {Singular form};
	\node [decision, below of=singular, node distance=2.5cm] (q4) {in quad-lexicon?};
	\node [block, below of=q4, node distance=2.3cm, text width=8em] (copy)  {Copy token};
	
	% Draw edges 
	\path [line] (input) -- (init);
	\path [line] (init) -- (q);
	\path [line] (q) -- (split);
	\path [line] (extract) -- (restore1);
	\path [line] (restore1) -- (end);
	\path [line] (q) -- node [near start, color=black] {yes} (extract);
	\path [line] (q) -- node [, color=black] {no}(split);
	\path [line] (split) -- (token);
	\path [line] (token) -- (q2);
	\path [line] (q2) -- node [near start, color=black] {\hspace*{2em} yes} (extract2);
	\path [line] (extract2) -- (restore2);
	\path [line] (restore2) -- (q3);
	\draw [line] (q3.east) -- ++(2cm,0cm)  |- node [near start, color=black] {yes~~} (token);
	\path [line] (q3) -- node [, color=black] {no}(restore3);
	\path [line] (restore3) -- (end2);
	\path [line] (q2) -- node [, color=black] {no}(singular);
	\path [line] (singular) -- (q4);
	\path [line] (q4.east) --  ++(0.7cm,0cm)  |- node [near start, color=black] {} (extract2);
	\path [line] (q4) -- node [, color=black] {no}(copy);
	\path [line] (copy.east) --  ++(1cm,0cm)  |- node [near start, color=black] {} (restore2);
    \end{tikzpicture}
    \caption{Flux diagram for the controlled term translation. If a complete term cannot be matched, a token by token translation is applied.}
    \label{fig:diagram}
\end{figure}

	\bigskip
	We apply the previous methodology to translate the CTs using two different lexicons: the multilingual MeSH (named {\tt MeSH} or {\tt M} in tables), and the union of the MeSH, Wikipedia, Apertium and manual multilingual lexicons ({\tt QuadLex} or {\tt Q}). For the final system (\emph{S2}) we also add the 5 million Wikidata entries. We cannot evaluate the quality of the translation because we do not have a subset of multilingual controlled terms other than MeSH itself, so we quantify the effect of our resources by the number of entries they are able to translate. Table~\ref{tab:tradsCT} shows the coverage for the CT$lan$L field in the languages of the project. In Table ~\ref{tab:tradsCTS2}, we show the improvements achieved by $S2$ with the new processing (Section~\ref{ss:cleaning}) and with the new resource (Section~\ref{ss:wikidata}). The MeSH thesaurus alone covers between 25\%-87\% of the all the controlled terms depending on the database and language. These numbers improve as we consider the coverage we can obtain by using also word-level mappings. %Our second system ($S2$) improves the coverage for all languages except for German, mainly because the cleaning we apply is thought for query translation. Therefore, for the final CLuBS controlled term translation we use $S1$ for German and $S2$ for English, French and Spanish.  
	The word-level mappings together with the usage of the extended lexicon allows us to reach almost a 100\% in all the cases. Results for CT$lan$H, IT$lan$H and IT$lan$L are shown in Appendix~\ref{ap:cts}.
	
	Even if at this point we cannot evaluate the quality of the translations, note that copying the source word into the output does not necessarily correspond to a wrong translation because in most cases the unknown words are named entities. Equivalently, using the quadrilingual lexicon to translate an entry does not assure a correct translation because, besides of the existing noise, the concatenation of word translations does not need to correspond to the term translation. However, we followed the proposed approach to maximize retrieval quality and not translation quality. 
	
	
	
	\begin{table}[t]
		\centering
		\flushleft{English}
		
		\small
		\begin{tabular}{llrrrrr}
			\toprule
			&       & \mc{2}{c}{Full descriptors, classes} & \mc{3}{c}{Tokens}\\
			\cmidrule(lr){3-4}   \cmidrule(lr){5-7}
			&Source & \mc{1}{c}{trad (\%)} & \mc{1}{c}{untrad (\%)} &\mc{1}{c}{trad (\%)} & \mc{1}{c}{untrad (\%)} & \mc{1}{c}{uniq}\\
			\midrule
			\multirow{5}{*}{\begin{sideways}MeSH\end{sideways}} 
			&ACCNO  &    344,453 (30.4\%)  &   787,342 (69.6\%)  & 1,325,648 (70.9\%)  & 545,113 (29.1\%) & 1834 \\
			&DFK    &    544,275 (33.3\%)  & 1,092,037 (66.7\%)  & 2,043,618 (77.2\%)  & 603,889 (22.8\%) & 2051 \\
			&NORART &      5,630 (24.6\%)  &    17,223 (75.4\%)  &    34,048 (86.9\%)  &   5,128 (13.1\%) &  86 \\
			&PDID   &        197 (43.9\%)  &       252 (56.1\%)  &       623 (80.5\%)  &     151 (19.5\%) &  87 \\
			&PMID   &  2,987,945 (86.9\%)  &   448,879 (13.1\%)  & 5,007,120 (97.1\%)  &  151,482 (2.9\%) & 242 \\
			\midrule
			\multirow{5}{*}{\begin{sideways}QuadLex\end{sideways}} 
			&ACCNO  &    586,440 (51.8\%)  & 545,355 (48.2\%)  & 1,861,278 (99.5\%)  & 9,483 (0.5\%) & 33 \\
			&DFK    &    900,396 (55.0\%)  & 735,916 (45.0\%)  & 2,640,589 (99.7\%)  & 6,918 (0.3\%) & 57 \\
			&NORART &      5,779 (25.3\%)  &  17,074 (74.7\%)  &    38,941 (99.4\%)  &   235 (0.6\%) &  9 \\
			&PDID   &        287 (63.9\%)  &     162 (36.1\%)  &       771 (99.6\%)  &     3 (0.4\%) &  1 \\
			&PMID   &  3,094,379 (90.0\%)  & 342,445 (10.0\%)  & 5,155,774 (99.9\%)  & 2,828 (0.1\%) & 30 \\
			\bottomrule
		\end{tabular}
		
		\flushleft{\normalsize German}
		\begin{tabular}{llrrrrr}
			\toprule
			&       & \mc{2}{c}{Full descriptors, classes} & \mc{3}{c}{Tokens}\\
			\cmidrule(lr){3-4}   \cmidrule(lr){5-7}
			&Source & \mc{1}{c}{trad (\%)} & \mc{1}{c}{untrad (\%)} &\mc{1}{c}{trad (\%)} & \mc{1}{c}{untrad (\%)} & \mc{1}{c}{uniq}\\
			\midrule
			\multirow{3}{*}{\begin{sideways}MeSH\end{sideways}} 
			&DFK    &   480,050 (29.1\%)  & 1,172,023 (70.9\%)  & 1,328,236 (61.7\%)  & 823,705 (38.3\%) & 3528 \\
			&PDID   &       182 (38.0\%)  &       297 (62.0\%)  &       425 (64.4\%)  &     235 (35.6\%) &  132 \\
			&PMID   & 2,915,784 (84.5\%)  &   535,085 (15.5\%)  & 4,321,857 (94.7\%)  & 240,222 (5.3\%)  &  160 \\
			\midrule
			\multirow{1}{*}{\begin{sideways}~QuadLex\end{sideways}} 
			&DFK     & 1,002,373 (60.7\%)  & 649,700 (39.3\%)  & 2,150,866 (100.0\%)  & 1,075 (0.0\%)  & 30 \\
			&PDID~~~~&       319 (66.6\%)  &     160 (33.4\%)  &       660 (100.0\%)  &     0 (0.0\%) &  0 \\
			&PMID    & 3,067,454 (88.9\%)  & 383,415 (11.1\%)  & 4,561,948 (100.0\%)  &   131 (0.0\%)  & 13 \\
			\bottomrule
		\end{tabular}
		
		
		\flushleft{\normalsize French}
		\begin{tabular}{llrrrrr}
			\toprule
			&       & \mc{2}{c}{Full descriptors, classes} & \mc{3}{c}{Tokens}\\
			\cmidrule(lr){3-4}   \cmidrule(lr){5-7}
			&Source & \mc{1}{c}{trad (\%)} & \mc{1}{c}{untrad (\%)} &\mc{1}{c}{trad (\%)} & \mc{1}{c}{untrad (\%)} & \mc{1}{c}{uniq}\\
			\midrule
			\multirow{1}{*}{\begin{sideways}M\end{sideways}} 
			&PMID~~~~&  2,520,288 (75.3\%)  & ~824,711 (24.7\%)  & ~~5,508,721 (92.9\%)  & ~419,105 (7.1\%)  & ~961 \\
			\midrule
			\multirow{-1}{*}{\begin{sideways}Q\end{sideways}} 
			&PMID     & 2,648,537 (79.2\%)  & 696,462 (20.8\%)  & 5,737,329 (96.8\%)  & 190,497 (3.2\%)  & 334 \\
			\bottomrule
		\end{tabular}
		\caption{Number of CT\emph{lan}L translated with the multilingual MeSH and the full QuadLexicon for English, German and French. There are no entries for Spanish. A CT term is splitted into two parts (the descriptor and the class specification), and in case of no-matching it is further splitted into tokens.}
		\label{tab:tradsCT}
	\end{table}

	
	
	
	\begin{table}[t]
		\centering
		\flushleft{English}
		
		\small
		\begin{tabular}{lrrrr}
			\toprule
			       & \mc{2}{c}{Full descriptors, classes} & \mc{2}{c}{Tokens}\\
			\cmidrule(lr){2-3}   \cmidrule(lr){4-5}
			Source & \mc{1}{c}{trad (\%)} & \mc{1}{c}{untrad (\%)} &\mc{1}{c}{trad (\%)} & \mc{1}{c}{untrad (\%)} \\
			\midrule

% 			\multirow{5}{*}{\begin{sideways}QuadLex\end{sideways}} 
			\emph{S1} ACCNO  &    586,440 (51.8\%)  & 545,355 (48.2\%)  & 1,861,278 (99.5\%)  & 9,483 (0.5\%) \\
			\emph{S2} ACCNO  &     598647 (52.9\%)  & 533148 (47.1\%)  & 1869206 (99.9\%)  & 1555 (0.1\%) \\
			\cmidrule(lr){2-5}
			\emph{S1} DFK    &    900,396 (55.0\%)  & 735,916 (45.0\%)  & 2,640,589 (99.7\%)  & 6,918 (0.3\%) \\
			\emph{S2} DFK    &    919275 (56.2\%)  & 717177 (43.8\%)  & 2646966 (100.0\%)  & 224 (0.0\%)  \\  \cmidrule(lr){2-5}
			\emph{S1} NORART &      5,779 (25.3\%)  &  17,074 (74.7\%)  &    38,941 (99.4\%)  &   235 (0.6\%) \\
			\emph{S2} NORART &      7,263 (31.8\%)  & 15,590 (68.2\%)  & 39,168 (100.0\%)  & 8 (0.0\%)     \\   \cmidrule(lr){2-5}
			\emph{S1} PDID   &        287 (63.9\%)  &     162 (36.1\%)  &       771 (99.6\%)  &     3 (0.4\%) \\
			\emph{S2} PDID   &        287 (63.9\%)  & 162 (36.1\%)  & 774 (100.0\%)  & 0 (0\%)      \\  \cmidrule(lr){2-5}
			\emph{S1} PMID   &  3,094,379 (90.0\%)  & 342,445 (10.0\%)  & 5,155,774 (99.9\%)  & 2,828 (0.1\%) \\
			\emph{S2} PMID   &  3,103,478 (90.2\%)  & 337,578 (9.8\%)  & 5,158,593 (100.0\%)  & 37 (0.0\%)\\
			\bottomrule
		\end{tabular}
		
		\flushleft{\normalsize German}
		\begin{tabular}{lrrrr}
			\toprule
			       & \mc{2}{c}{Full descriptors, classes} & \mc{2}{c}{Tokens}\\
			\cmidrule(lr){2-3}   \cmidrule(lr){4-5}
			Source & \mc{1}{c}{trad (\%)} & \mc{1}{c}{untrad (\%)} &\mc{1}{c}{trad (\%)} & \mc{1}{c}{untrad (\%)} \\
			\midrule
% 			\multirow{1}{*}{\begin{sideways}~QuadLex\end{sideways}} 
			\emph{S1} DFK     & 1,002,373 (60.7\%)  & 649,700 (39.3\%)  & 2,150,866 (100.0\%)  & 1,075 (0.0\%) \\
			\emph{S2} DFK     & 1,010,788 (61.2\%)  & 641,286 (38.8\%)  & 2,150,863  (99.9\%)  & 1,078 (0.1\%)  \\ \cmidrule(lr){2-5}
			\emph{S1} PDID~~~~&      319 (66.6\%)  &     160 (33.4\%)  &       660 (100.0\%)  &     0 (0.0\%) \\
			\emph{S2} PDID~~~~&      318 (66.4\%)  & 161 (33.6\%)  & 658 (99.7\%)  & 2 (0.3\%) 
			\\ \cmidrule(lr){2-5}
			\emph{S1} PMID    & 3,067,454 (88.9\%)  & 383,415 (11.1\%)  & 4,561,948 (100.0\%)  &   131 (0.0\%) \\
			\emph{S2} PMID    & 3,054,860 (88.5\%)  & 397,321 (11.5\%)  & 4,556,463 (99.8\%)  & 7352 (0.2\%)\\
			\bottomrule
		\end{tabular}
		
		
		\flushleft{\normalsize French}
		\begin{tabular}{lrrrr}
			\toprule
			       & \mc{2}{c}{Full descriptors, classes} & \mc{2}{c}{Tokens}\\
			\cmidrule(lr){2-3}   \cmidrule(lr){4-5}
			Source & \mc{1}{c}{trad (\%)} & \mc{1}{c}{untrad (\%)} &\mc{1}{c}{trad (\%)} & \mc{1}{c}{untrad (\%)}\\
			\midrule
% 			\multirow{-1}{*}{\begin{sideways}Q\end{sideways}} 
			\emph{S1} PMID     & 2,648,537 (79.2\%)  & 696,462 (20.8\%)  & 5,737,329 (96.8\%)  & 190,497 (3.2\%)  \\
			\emph{S2} PMID     & 2,982,535 (87.1\%)  & 442,789 (12.9\%)  & 5,670,349 (98.6\%)  &  77,790 (1.4\%)  \\
			\bottomrule
		\end{tabular}
		\caption{Number of CT\emph{lan}L translated with the multilingual MeSH and the full QuadLexicon of System 2 for English, German and French. There are no entries for Spanish. }
		\label{tab:tradsCTS2}
	\end{table}
	\subsection{Software}
	
	A python script takes care of the CT translation. It can be found in the {\tt DBtranslator} package%
	\footnote{\url{}https://github.com/clubs-project/DBtranslator}
	together with all the software developed to translate the different components of the PubPsych database.
	The complete translation pipeline going from downloading the field data for all the documents in the database, to translate them and uploading the translations is run by {\tt tradCTs.sh}:
	
	\begin{verbatim}
	user@machine:~/home/DBtranslator/scripts/$  bash tradCTs.sh -h
	tradCTs.sh -f CTH|CTL|ITH|ITL [-h] 
	
	where:
	-h  show this help text
	-f  field to translate [CTH|CTL|ITH|ITL]
	
	Example:
	bash tradCTs.sh -f CTH
	\end{verbatim}
	
	If you want to consider a new field, add it to {\tt preproField4trad.py}. If you want to use the script only on a subset of the database, please, modify the Solr query accordingly in the same file. 
	
	The up-to-date instructions for installing and using the software can be found in the git repository:
	
	\url{https://github.com/clubs-project/DBtranslator}
	
	
	\section{Query Translation}
	\label{s:qtrad}
	
	\subsection{Methodology}
	
	
\tikzstyle{io} = [rectangle, draw=black, fill=white, rounded corners, text width=12em, text badly centered, inner sep=8pt, node distance=2.5cm]
\tikzstyle{decision} = [diamond, draw, fill=white, text width=5em, text badly centered, node distance=3.5cm, inner sep=0pt]
\tikzstyle{block} = [rectangle, draw=black, fill=white, node distance=2.5cm, text width=14em, text badly centered, minimum height=2.5em, inner sep=0pt]
\tikzstyle{line} = [draw,->,-latex,black,line width=1pt]
\begin{figure}[t!]
    \centering
    \begin{tikzpicture}[auto,draw=black,line width=1pt,scale=0.6,transform shape]
    \tikzstyle{every node}=[font=\large]
	% Place nodes
	\node [io] (input) {INPUT: Parsed term\\ \emph{action potentials}};
	%\node [block, below of=input, text width=26em, minimum height=3.5em, node distance=2cm] (init) {Generation of term alternatives\\ (\emph{Action Potentials}, \emph{action potentials}, \emph{Action potentials}) };
	\node [decision, below of=input, node distance=2.5cm] (q) {in quad-lexicon?};
	\node [block, left of=q, node distance=6cm, text width=20em, minimum height=5em, inner sep=5pt] (extract)  {Extract entry\\ \emph{action potentials $|||$ es:potenciales de acción $|||$ de:aktionspotentiale $|||$ fr:potentiels d'action}};
	%\node [block, below of=extract, node distance=2.35cm] (restore1) {Restore capitalisation};
	\node [io, below of=extract, node distance=2.6cm] (end) {OUTPUT: Quad-term};
	
	\node [block, below of=q, node distance=2.4cm] (split) {Split by tokens};
	\node [block, below of=split, node distance=1.8cm] (token) {Token alternatives};
	\node [decision, below of=token, node distance=2.5cm] (q2) {in quad-lexicon?};
	\node [block, right of=q2, node distance=5cm, text width=9em] (extract2)  {Extract entry};
	\node [decision, below of=extract2, node distance=2.6cm] (q3) {more tokens?};
	\node [block, below of=q3, node distance=2.2cm] (restore3) {Recompose full entry};
	\node [io, below of=restore3, node distance=2.4cm] (end2) {OUTPUT: Quad-term};
	\node [block, below of=q2, node distance=2.3cm, text width=9em] (singular)  {Singular form};
	\node [decision, below of=singular, node distance=2.5cm] (q4) {in quad-lexicon?};
	\node [block, below of=q4, node distance=2.3cm, text width=9em] (copy)  {Copy token};
	
	% Draw edges 
	\path [line] (input) -- (q);
	\path [line] (q) -- (split);
	\path [line] (extract) -- (end);
	\path [line] (q) -- node [near start, color=black] {yes} (extract);
	\path [line] (q) -- node [, color=black] {no}(split);
	\path [line] (split) -- (token);
	\path [line] (token) -- (q2);
	\path [line] (q2) -- node [near start, color=black] {\hspace*{2em} yes} (extract2);
	\path [line] (extract2) -- (q3);
	\draw [line] (q3.east) -- ++(2cm,0cm)  |- node [near start, color=black] {yes~~} (token);
	\path [line] (q3) -- node [, color=black] {no}(restore3);
	\path [line] (restore3) -- (end2);
	\path [line] (q2) -- node [, color=black] {no}(singular);
	\path [line] (singular) -- (q4);
	\path [line] (q4.east) --  ++(0.7cm,0cm)  |- node [near start, color=black] {} (extract2);
	\path [line] (q4) -- node [, color=black] {no}(copy);
	\path [line] (copy.east) --  ++(1cm,0cm)  |- node [near start, color=black] {} (q3);
    \end{tikzpicture}
    \caption{Flowchart for query term translation. If a complete term cannot be matched, a token by token translation is applied.}
    \vspace{-6mm}
    \label{fig:diagram}
\end{figure}

	
	\subsection{Off-line Software}

	\subsection{Online Integration into PubPsych}
	\label{ss:integration}
	The approach has been implemented with respect to Solr 6.6.5. We have added four classes to the existing PubPsych backend:
	\begin{itemize}
		\item \textbf{QueryFieldRewriter}: This is the main class of the translation module where the actual translation takes place. In a \textit{QueryFieldRewriter} object, the two dictionaries (high-quality and low-quality) and a mapping of field names to language-specific field names are stored. We map field names to language-specific field names, because if a language-specific version of a field exists, we can directly query the correct field with the translation. Since the database schema does not follow a strict pattern in the mapping of these language-specific field names, we need to store the mapping.
		\item \textbf{QueryNode}: The purpose of the \textit{QueryNode} class is to provide an interface to manipulate the queries more easily than with Solr’s \textit{Query} class. The main advantage is that one only has to deal with one class for all query objects (and not with the many subclasses of \textit{Query} that all provide different methods/fields). This way, one does not have to build new \textit{Query} objects for each change, but change the respective \textit{QueryNode} object instead and then map it back to a \textit{Query} object once all changes due to translation have been performed.
		\item \textbf{Translation}: The \textit{Translation} class is used to store information about the translation of a specific field and string. More precisely, we store the original field name and a mapping of language-specific field names to translation strings in the respective target language. For some fields (e.g. \textit{text}), there are no language-specific field names. In this case, we have to use dummy field names that do not exist in the database schema to have different keys for the different languages. We store in an additional map whether a field name actually exists in the schema in order to search only existing fields in the final query.
		\item \textbf{PreTranslationInfo}: This classed is used to store information on the query before the actual translation happens. We concatenate the strings of all subqueries that fulfill certain conditions and store to which subqueries the concatenation belongs. 
	\end{itemize}

	For more details on fields and methods of the classes, see \url{https://github.com/clubs-project/documentation/blob/master/sofware/onlineQueryTranslation.html}.
	
	\subsection{Evaluation of the Queries}
	\label{ss:evaluation}
	We have gathered approximately 100 real-life queries in each of the languages of the PubPsych database (English, German, French and Spanish) and another 100 real-life queries whose language could not be classified. These queries are the same as in \url{https://link.springer.com/chapter/10.1007/978-3-030-14401-2_4}.
	
	The experiments were run on a machine with 96 cores at 2.4 gigahertz and 1 terabyte memory, but processes were not parallelized and the Java VM was only given 10 gigabyte memory each time a new Solr instance was initialized (which had to be done for every collection of 100 queries to reset the statistics, see below). We had seven different settings. For each setting, the JAR files had to built anew, and we let the software translate all 500 queries in each setting. Running a script that performed all these actions took a bit more than half an hour. 
	
	In each of the seven settings, we have used a different combination of dictionaries. If not mentioned otherwise, the dictionaries were used in their ``non-diff'' version, which means that entries that are the same in all languages were kept (e.g. ``Ich bin ein Berliner'' is translated in all languages as ``Ich bin ein Berliner'' since it is the title of a Wikipedia article), whereas they were deleted in the ``diff'' version. In addition to the MeSh dictionary and the quad-lexicon described in section \ref{ss:cleaning} we have built another dictionary by extracting entities in all the Wikidata dump that exist simultaneously in the four languages.
	
	\begin{enumerate}
		\item High-quality dictionary: MeSh, low-quality dictionary: the concatenation of the quad-lexicon from section 2.5 and the Wikidata dictionary
		\item High-quality dictionary: none, low-quality dictionary: Wikidata dictionary
		\item High-quality dictionary: none, low-quality dictionary: Wikidata dictionary ("diff" version)
		\item High-quality dictionary: none, low-quality dictionary: quad-lexicon from section 2.5
		\item High-quality dictionary: none, low-quality dictionary: quad-lexicon from section 2.5 ("diff" version)
		\item High-quality dictionary: MeSh, low-quality dictionary: none
		\item High-quality dictionary: MeSh, low-quality dictionary: quad-lexicon from section 2.5
	\end{enumerate}
	
	We do not only evaluate the actual translations of the queries, but have also acquired some statistics for each combination of query collection and setting. The numbers collected are the following:
	\begin{itemize}
		\item \textbf{MeSh usage word level} (muw): Number of words that could be translated with the MeSh dictionary (inside \textit{translate}). This number is incremented when a string that either is the result of a concatenation of subqueries or a multi-token phrase is split into tokens (because it could not be translated as a whole) and a token or a possible singular form of it could be translated using the MeSh dictionary. If the MeSh dictionary is not used in a setting, this number is 0. 
		\item \textbf{MeSh usage multi-token level} (mum): Number of whole multi-token strings that could be translated with the MeSh dictionary. The \textit{multi-token strings} include concatenations of several subqueries (according to the criteria mentioned in the detailed implementation explanation) and multi-word phrases which were explicitly marked as a phrase by the user. If a concatenation cannot be translated as a whole (meaning a single lookup of the whole string in the dictionary), but all its tokens can be translated using only the MeSh dictionary, this number \textbf{is not} incremented. If a phrase cannot be translated as whole, but all its tokens can be translated using only the MeSh dictionary, this number \textbf{is} incremented. The reasoning behind this difference in counting is that if a user explicitly marks several tokens as a phrase, it is likely that these tokens actually form a phrase, whereas the simple concatenation of all tokens in a query does not take into account any semantic information.
		
		If the MeSh dictionary is not used in a setting, this number is 0. 
		\item \textbf{MeSh usage query level} (muq): Number of queries that could entirely (including all subqueries) be translated using only the MeSh dictionary. If a token could not be translated, but a possible singular form of it and this singular form was found in MeSh, then the whole query would still count as ``translated only with MeSh'' (assuming that everything else or a singular form of each token was found in MeSh). If the MeSh dictionary is not used in a setting, this number is 0. 
		\item \textbf{Backoff usage word level} (buw): Number of words that could be translated with the low-quality dictionary. The conditions are the same as for MeSh usage word level. If no low-quality dictionary is used in a setting, this number is 0.
		\item \textbf{Backoff usage multi-token level} (bum): Number of whole multi-token strings that could be translated with the low-quality dictionary. The conditions are the same as for MeSh usage multi-token level. If no low-quality dictionary is used in a setting, this number is 0.
		\item \textbf{Backoff usage query level} (buq): Number of queries that could entirely (including all subqueries) be translated only using the low-quality dictionary. This implies that none of the strings or tokens could be found in the MeSh dictionary, since we always try that one first. The conditions are the same as for MeSh usage query level. If no low-quality dictionary is used in a setting, this number is 0.
		\item \textbf{Copies at word level} (cw): Number of words that could not be translated, not even a possible singular form.
		\item \textbf{Copies at multi-token level} (cm): Number of whole multi-token strings where nothing could be translated, not even a possible singular form (cf. MeSh usage string level for what "whole multi-token string" refers to).		
		\item \textbf{Copies at query level} (cq): Number of queries where nothing could be translated, not even a possible singular form.
		\item \textbf{Singular usage word level} (suw): Number of words that could not be translated in their original form, but a possible singular form could be translated.
		\item \textbf{Singular usage multi-token level} (sum): Number of phrases where at least one token could not be translated, but a possible singular form of it. Note that this excludes concatenations and that the counting is different to other numbers on the multi-token level (not all tokens of the phrase have to be translated using singular forms).
		\item \textbf{Singular usage query level} (suq): Number of queries that were entirely translated only using possible singular forms. This also means that no token was copied at all.
	\end{itemize}


	\begin{figure}[h]
		\begin{tabular}{c|c|c|c|c|c|c|c|c|c|c|c|c|c}
			& muw & mum & muq & buw & bum & buq & cw & cm & cq & suw & sum & suq \\
			\hline
1e & 102 & 1 & 16 & 161 & 6 & 24 & 34 & 0 & 0 & 2 & 0 & 0\\
1d & 49 & 1 & 10 & 106 & 0 & 21 & 72 & 0 & 16 & 8 & 0 & 0\\
1s & 103 & 5 & 17 & 183 & 2 & 19 & 53 & 0 & 1 & 22 & 0 & 0\\
1f & 85 & 8 & 13 & 185 & 2 & 14 & 65 & 0 & 2 & 18 & 0 & 0\\
1n & 13 & 0 & 8 & 119 & 4 & 58 & 31 & 0 & 17 & 0 & 0 & 0\\
2e & 0 & 0 & 0 & 231 & 6 & 54 & 67 & 0 & 2 & 7 & 0 & 0\\
2d & 0 & 0 & 0 & 116 & 0 & 24 & 112 & 0 & 29 & 5 & 0 & 0\\
2s & 0 & 0 & 0 & 258 & 4 & 52 & 87 & 0 & 2 & 45 & 0 & 3\\
2f & 0 & 0 & 0 & 247 & 7 & 40 & 94 & 0 & 4 & 22 & 0 & 1\\
2n & 0 & 0 & 0 & 128 & 4 & 68 & 34 & 0 & 19 & 0 & 0 & 0\\
3e & 0 & 0 & 0 & 178 & 5 & 33 & 120 & 0 & 8 & 11 & 0 & 0\\
3d & 0 & 0 & 0 & 92 & 0 & 22 & 134 & 0 & 34 & 9 & 0 & 0\\
3s & 0 & 0 & 0 & 184 & 4 & 27 & 159 & 0 & 5 & 37 & 0 & 3\\
3f & 0 & 0 & 0 & 173 & 6 & 20 & 169 & 0 & 10 & 21 & 0 & 1\\
3n & 0 & 0 & 0 & 52 & 2 & 13 & 112 & 0 & 58 & 1 & 0 & 0\\
4e & 0 & 0 & 0 & 230 & 2 & 52 & 77 & 0 & 1 & 6 & 0 & 0\\
4d & 0 & 0 & 0 & 117 & 0 & 32 & 110 & 0 & 29 & 10 & 0 & 1\\
4s & 0 & 0 & 0 & 205 & 2 & 32 & 144 & 0 & 3 & 24 & 0 & 2\\
4f & 0 & 0 & 0 & 193 & 4 & 29 & 156 & 0 & 4 & 20 & 0 & 0\\
4n & 0 & 0 & 0 & 23 & 0 & 5 & 147 & 0 & 84 & 2 & 0 & 0\\
5e & 0 & 0 & 0 & 230 & 2 & 52 & 77 & 0 & 1 & 6 & 0 & 0\\
5d & 0 & 0 & 0 & 117 & 0 & 32 & 110 & 0 & 29 & 10 & 0 & 1\\
5s & 0 & 0 & 0 & 205 & 2 & 32 & 144 & 0 & 3 & 24 & 0 & 2\\
5f & 0 & 0 & 0 & 193 & 4 & 29 & 156 & 0 & 4 & 20 & 0 & 0\\
5n & 0 & 0 & 0 & 23 & 0 & 5 & 147 & 0 & 84 & 2 & 0 & 0\\
6e & 106 & 1 & 16 & 0 & 0 & 0 & 201 & 0 & 27 & 0 & 0 & 0\\
6d & 52 & 1 & 10 & 0 & 0 & 0 & 171 & 0 & 55 & 3 & 0 & 0\\
6s & 106 & 5 & 17 & 0 & 0 & 0 & 233 & 0 & 25 & 7 & 0 & 0\\
6f & 89 & 8 & 14 & 0 & 0 & 0 & 247 & 0 & 28 & 3 & 0 & 1\\
6n & 14 & 0 & 8 & 0 & 0 & 0 & 154 & 0 & 87 & 1 & 0 & 0\\
7e & 105 & 1 & 16 & 135 & 2 & 12 & 65 & 0 & 1 & 4 & 0 & 0\\
7d & 50 & 1 & 10 & 80 & 0 & 16 & 95 & 0 & 19 & 9 & 0 & 0\\
7s & 103 & 5 & 17 & 108 & 2 & 15 & 126 & 0 & 1 & 17 & 0 & 1\\
7f & 86 & 8 & 13 & 110 & 1 & 7 & 140 & 0 & 4 & 10 & 0 & 0\\
7n & 13 & 0 & 8 & 37 & 2 & 8 & 116 & 0 & 62 & 3 & 0 & 2\\
		\end{tabular}
		\caption{The number refers to the setting (see beginning of section \ref{ss:evaluation}), the letter indicates the set of queries that has been translated (\textit{e}: English, \textit{d}: German, \textit{s}: Spanish, \textit{f}: French and \textit{n}: language could not be classified).}
		\label{fig:eval}
	\end{figure}
	
	\section{Conclusions}
	\label{s:conclusions}
	
	Usage of indexing terms, either controlled or uncontrolled, differs vastly between different database segments contained in PubPsych. By just using the simplest mapping approach of the quadrilingual MeSH thesaurus, we were able to map between 30\%--75\% of all controlled terms and 9\%--40\% of free indexing terms between the four languages. A more refined mapping approach and an out-of-domain extension of the quadrilingual lexicon resulted in increased mapping of 67\%--100\% (controlled terms) and 62\%--94\% (free terms) respectively. We did not check actual translation quality, but just coverage. The worst mapping performance in all scenarios was exhibited with the PSYNDEX database segment, which uses controlled terminology from the thesaurus of the American Psychological Association. We did not include data from that thesaurus into this evaluation, because it is not available in all for languages.
	
	%
	% ---- Appendeces ----
	%
	\appendix
	\section{Basic Rules for Plural Formation}
	\label{ap:plural}
	
	In order to obtain the a possible singular form of unseen tokens we apply the following basic rules:
	
	\renewcommand{\labelitemi}{$\star$}
	\paragraph{English}
	\begin{itemize}
		\itemsep-0.2em 
		\item NOUN-y $\Leftarrow$ NOUN-ies
		\item NOUN $\Leftarrow$ NOUN-es
		\item NOUN $\Leftarrow$ NOUN-s
	\end{itemize}
	
	\paragraph{French}
	\begin{itemize}
		\itemsep-0.2em 
		\item NOUN $\Leftarrow$ NOUN-s
	\end{itemize}
	
	\paragraph{German}
	\begin{itemize}
		\itemsep-0.2em 
		\item NOUN (\"-) $\Leftarrow$ NOUN-er
		\item NOUN $\Leftarrow$ NOUN-n
		\item NOUN $\Leftarrow$ NOUN-e
		\item NOUN $\Leftarrow$ NOUN-s
	\end{itemize}
	
	\paragraph{Spanish}
	\begin{itemize}
		\itemsep-0.2em 
		\item NOUN $\Leftarrow$ NOUN-es
		\item NOUN $\Leftarrow$ NOUN-s
	\end{itemize}
	
	
	% CODE
	%    # for Spanish
	%     if len(singular)>1 and language=="es":
	%        if singular[:-2] == 'es': 
	%           singular = singular[:-2]
	%        elif singular[-1] == 's': 
	%           singular = singular[:-1]
	%     # for French
	%     if len(singular)>1 and language=="fr":
	%        if singular[-1] == 's': 
	%           singular = singular[:-1]
	%     # for English
	%     if len(singular)>3 and language=="en":
	%        if singular[:-3] == 'ies': 
	%           return singular[:-3]+"y"
	%     if len(singular)>2 and language=="en":
	%        if singular[:-2] == 'es': 
	%           return singular[:-2]
	%     if len(singular)>1 and language=="en":
	%        if singular[-1] == 's': 
	%           singular = singular[:-1]
	%     # for German
	%     if len(singular)>2 and language=="de":
	%        if singular[:-2] == 'er': 
	%           return remove_diacritic(singular[:-2]).decode() #we need to remove the umlaut too
	%        if singular[-1] == 'n': 
	%           singular = singular[:-1]
	%     if len(singular)>1 and language=="de":
	%        if singular[-1] == 'e' or  singular[-1] == 's': 
	%           singular = singular[:-1]
	
% 	\newpage
	\section{Translation Coverage for CT\mbox{\boldmath $lan$}H, IT\mbox{\boldmath $lan$}H and IT\mbox{\boldmath $lan$}L}
	\label{ap:cts}

	\begin{table}[h]
		\centering
		\flushleft{English}
		
		\small
		\begin{tabular}{llrrrr}
			\toprule
			&       & \mc{2}{c}{Full descriptors, classes} & \mc{2}{c}{Tokens}\\
			\cmidrule(lr){3-4}   \cmidrule(lr){5-6}
			&Source & \mc{1}{c}{trad (\%)} & \mc{1}{c}{untrad (\%)} &\mc{1}{c}{trad (\%)} & \mc{1}{c}{untrad (\%)} \\
			\midrule
			\multirow{3}{*}{\begin{sideways}M\end{sideways}} 
			&ACCNO  & 104,815 (30.2\%)  & 241,866 (69.8\%)  &   426,013 (76.1\%)  & 133,654 (23.9\%) \\
			&DFK    & 462,302 (35.5\%)  & 838,814 (64.5\%)  & 1,670,304 (81.0\%)  & 391,538 (19.0\%) \\
			&PMID   & 699,993 (74.3\%)  & 242,666 (25.7\%)  & 1,423,324 (99.4\%)  &   8,840 (0.6\%)  \\
			\midrule
			\multirow{3}{*}{\begin{sideways}Q\end{sideways}} 
			&ACCNO  & 162,867 (47.0\%)  & 183,814 (53.0\%)  &   557,624 (99.6\%)  & 2,043 (0.4\%)  \\
			&DFK    & 694,203 (53.4\%)  & 606,913 (46.6\%)  & 2,056,459 (99.7\%)  & 5,383 (0.3\%) \\
			&PMID   & 705,345 (74.8\%)  & 237,314 (25.2\%)  & 1,430,707 (99.9\%)  & 1,457 (0.1\%)   \\
			\bottomrule
		\end{tabular}
		
		\flushleft{\normalsize German}
		\begin{tabular}{llrrrr}
			\toprule
			&       & \mc{2}{c}{Full descriptors, classes} & \mc{2}{c}{Tokens}\\
			\cmidrule(lr){3-4}   \cmidrule(lr){5-6}
			&Source & \mc{1}{c}{trad (\%)} & \mc{1}{c}{untrad (\%)} &\mc{1}{c}{trad (\%)} & \mc{1}{c}{untrad (\%)}\\
			\midrule
			\multirow{2}{*}{\begin{sideways}M\end{sideways}} 
			&DFK    & 393,698 (30.0\%)  & 916,459 (70.0\%)  & 1,123,046 (66.5\%)  & 565,367 (33.5\%) \\
			&PMID~~ & 701,120 (73.5\%)  & 252,731 (26.5\%)  & 1,273,245 (99.0\%)  & 12,373 (1.0\%) \\
			\midrule
			\multirow{2}{*}{\begin{sideways}Q\end{sideways}} 
			&DFK    & 747,809 (57.1\%)  & 562,348 (42.9\%)  & 1,686,612  (99.9\%)  & 1,801 (0.1\%) \\
			&PMID   & 708,180 (74.2\%)  & 245,671 (25.8\%)  & 1,285,596 (100.0\%)  & 22 (0.0\%) \\
 			\bottomrule
		\end{tabular}
		
		
		\flushleft{\normalsize French}
		\begin{tabular}{llrrrr}
			\toprule
			&       & \mc{2}{c}{Full descriptors, classes} & \mc{2}{c}{Tokens}\\
			\cmidrule(lr){3-4}   \cmidrule(lr){5-6}
			&Source & \mc{1}{c}{trad (\%)} & \mc{1}{c}{untrad (\%)} &\mc{1}{c}{trad (\%)} & \mc{1}{c}{untrad (\%)} \\
			\midrule
			\multirow{1}{*}{\begin{sideways}M\end{sideways}} 
			&PMID~~~~~& 649,818 (70.0\%)  & 278,327 (30.0\%)  & 1,572,917 (97.6\%)  & ~38,389 (2.4\%) \\
			\midrule
			\multirow{-1}{*}{\begin{sideways}Q\end{sideways}} 
			&PMID    & 665,293 (71.7\%)  & 262,852 (28.3\%)  & 1,603,726 (99.5\%)  & 7,580 (0.5\%) \\
			\bottomrule
		\end{tabular}
		\caption{Number of CT\emph{lan}H translated with the multilingual MeSH (M) and the full QuadLexicon (Q) for English, German and French. There are no entries for Spanish.}
		\label{tab:tradsCTH}
	\end{table}


	\begin{table}[h]
		\centering
		\flushleft{English}
		
		\small
		\begin{tabular}{llrrrr}
			\toprule
			&       & \mc{2}{c}{Full descriptors, classes} & \mc{2}{c}{Tokens}\\
			\cmidrule(lr){3-4}   \cmidrule(lr){5-6}
			&Source & \mc{1}{c}{trad (\%)} & \mc{1}{c}{untrad (\%)} &\mc{1}{c}{trad (\%)} & \mc{1}{c}{untrad (\%)} \\
			\midrule
			\multirow{5}{*}{\begin{sideways}MeSH\end{sideways}}
			&ACCNO  & 34,613 (26.0\%)  &  98,309 (74.0\%)  & 176,057 (80.4\%)  &  42,829 (19.6\%)\\
			&DFK    &  3,017  (9.1\%)  &  30,288 (90.9\%)  &  35,389 (73.1\%)  &  13,041 (26.9\%)\\
			&NBN    & 42,314 (20.8\%)  & 161,394 (79.2\%)  & 199,607 (61.6\%)  & 124,180 (38.4\%) \\
			&PMID   & 39,300 (35.4\%)  &  71,565 (64.6\%)  & 126,201 (79.4\%)  &  32,820 (20.6\%) \\
			&POID   &    818 (14.3\%)  &   4,906 (85.7\%)  &   5,687 (61.5\%)  &   3,553 (38.5\%) \\
			\midrule
			\multirow{5}{*}{\begin{sideways}QuadLex\end{sideways}} 
			&ACCNO  & 44,560 (33.5\%)  &  88,362 (66.5\%)  & 204,644 (93.5\%)  & 14,242 (6.5\%)   \\
			&DFK    &  6,628 (19.9\%)  &  26,677 (80.1\%)  &  44,678 (92.3\%)  &  3,752 (7.7\%)   \\
			&NBN    & 74,130 (36.4\%)  & 129,578 (63.6\%)  & 259,977 (80.3\%)  & 63,810 (19.7\%) \\
			&PMID   & 44,686 (40.3\%)  &  66,179 (59.7\%)  & 146,631 (92.2\%)  & 12,390 (7.8\%)  \\
			&POID   &  1,878 (32.8\%)  &   3,846 (67.2\%)  &   8,117 (87.8\%)  &  1,123 (12.2\%) \\
			\bottomrule
		\end{tabular}
		
		\flushleft{\normalsize German} \\
		\begin{tabular}{llrrrr}
			\toprule
			&       & \mc{2}{c}{Full descriptors, classes} & \mc{2}{c}{Tokens}\\
			\cmidrule(lr){3-4}   \cmidrule(lr){5-6}
			&Source & \mc{1}{c}{trad (\%)} & \mc{1}{c}{untrad (\%)} &\mc{1}{c}{trad (\%)} & \mc{1}{c}{untrad (\%)}\\
			\midrule
			\multirow{1}{*}{\begin{sideways}M\end{sideways}} 
			&DFK~~~~~~~  &  2,246 (6.8\%)   & 30,890 (93.2\%)  & ~~22,254 (58.7\%)  & 15,640 (41.3\%) \\
			\midrule
			\multirow{1}{*}{\begin{sideways}Q\end{sideways}} 
			&DFK          & 6,415 (19.4\%)  & 26,721 (80.6\%)  & ~~29,861 (78.8\%)  & 8,033 (21.2\%) \\
			\bottomrule
		\end{tabular}		
		\caption{Number of IT\emph{lan}H translated with the multilingual MeSH (M) and the full QuadLexicon (Q) for English and German. There are no entries for Spanish or French.}
		\label{tab:tradsITH}
	\end{table}

	
	\begin{table}[h]
		\centering
		\flushleft{English}
		
		\small
		\begin{tabular}{llrrrr}
			\toprule
			&       & \mc{2}{c}{Full descriptors, classes} & \mc{2}{c}{Tokens}\\
			\cmidrule(lr){3-4}   \cmidrule(lr){5-6}
			&Source & \mc{1}{c}{trad (\%)} & \mc{1}{c}{untrad (\%)} &\mc{1}{c}{trad (\%)} & \mc{1}{c}{untrad (\%)} \\
			\midrule
			\multirow{5}{*}{\begin{sideways}MeSH\end{sideways}} 
			&ACCNO  &        0    (0\%)   &    62,937 (100.0\%) &   69,785 (64.5\%)  &   38,327 (35.5\%) \\
			&DFK    &     1,035 (10.0\%)  &     9,305 (90.0\%)  &   10,316 (71.2\%)  &    4,171 (28.8\%) \\
			&INIST  & 1,084,844 (34.7\%)  & 2,042,320 (65.3\%)  & 3,208,156 (67.6\%) & 1,537,860 (32.4\%) \\
			&NORART &     3,740 (21.4\%)  &    13,719 (78.6\%)  &   18,554 (70.4\%)  &    7,802 (29.6\%)  \\
			&PMID   &    36,784 (24.2\%)  &   115,237 (75.8\%)  &  173,377 (68.8\%)  &   78,807 (31.2\%)  \\
			\midrule
			\multirow{5}{*}{\begin{sideways}QuadLex\end{sideways}} 
			&ACCNO  &    16,682 (26.5\%)  &    46,255 (73.5\%)  &   108,112 (100.0\%) &      0 (0\%) \\
			&DFK    &     2,152 (20.8\%)  &     8,188 (79.2\%)  &    13,419 (92.6\%)  &   1,068 (7.4\%) \\
			&INIST  & 1,716,205 (54.9\%)  & 1,410,959 (45.1\%)  & 4,389,674 (92.5\%)  & 356,342 (7.5\%)   \\
			&NORART &     6,734 (38.6\%)  &    10,725 (61.4\%)  &    24,382 (92.5\%)  &   1,974 (7.5\%) \\
			&PMID   &    59,711 (39.3\%)  &    92,310 (60.7\%)  &   228,096 (90.4\%)  &  24,088 (9.6\%)   \\
			\bottomrule
		\end{tabular}
		
		\flushleft{\normalsize German} \\
		\begin{tabular}{llrrrr}
			\toprule
			&       & \mc{2}{c}{Full descriptors, classes} & \mc{2}{c}{Tokens}\\
			\cmidrule(lr){3-4}   \cmidrule(lr){5-6}
			&Source & \mc{1}{c}{trad (\%)} & \mc{1}{c}{untrad (\%)} &\mc{1}{c}{trad (\%)} & \mc{1}{c}{untrad (\%)}\\
			\midrule
			\multirow{2}{*}{\begin{sideways}M\end{sideways}} 
			&DFK    &  450 (4.5\%)  & 9,626 (95.5\%)  & 6,272 (55.3\%)  & 5,075 (44.7\%)  \\
			&INIST  &  1 (6.2\%)  & 15 (93.8\%)  & 7 (35.0\%)  & 13 (65.0\%) \\
			\midrule
			\multirow{2}{*}{\begin{sideways}Q\end{sideways}} 
			&DFK    &  1,723 (17.1\%)  & 8,353 (82.9\%)  & 8,614 (75.9\%)  & 2,733 (24.1\%)  \\
			&INIST  &  2 (12.5\%)  & 14 (87.5\%)  & 10 (50.0\%)  & 10 (50.0\%) \\
			\bottomrule
		\end{tabular}		
		
		\flushleft{\normalsize French} \\
		\begin{tabular}{llrrrr}
			\toprule
			&       & \mc{2}{c}{Full descriptors, classes} & \mc{2}{c}{Tokens}\\
			\cmidrule(lr){3-4}   \cmidrule(lr){5-6}
			&Source & \mc{1}{c}{trad (\%)} & \mc{1}{c}{untrad (\%)} &\mc{1}{c}{trad (\%)} & \mc{1}{c}{untrad (\%)} \\
			\midrule
			\multirow{1}{*}{\begin{sideways}M\end{sideways}} 
			&INIST  & 1,186,988 (40.9\%)  & 1,713,817 (59.1\%)  & 3,214,926 (71.7\%)  & 1,268,994 (28.3\%) \\
			\midrule
			\multirow{1}{*}{\begin{sideways}Q\end{sideways}} 
			&INIST  & 1,618,921 (55.8\%)  & 1,281,884 (44.2\%)  & 4,077,461 (90.9\%)  & 406,459 (9.1\%) \\
			\bottomrule
		\end{tabular}

		\flushleft{\normalsize Spanish}
		\begin{tabular}{llrrrr}
			\toprule
			&       & \mc{2}{c}{Full descriptors, classes} & \mc{2}{c}{Tokens}\\
			\cmidrule(lr){3-4}   \cmidrule(lr){5-6}
			&Source & \mc{1}{c}{trad (\%)} & \mc{1}{c}{untrad (\%)} &\mc{1}{c}{trad (\%)} & \mc{1}{c}{untrad (\%)} \\
			\midrule
			\multirow{2}{*}{\begin{sideways}M\end{sideways}} 
			&INIST  & 896,891 (32.6\%)  & 1,851,974 (67.4\%)  & 2,784,473 (67.3\%)  & 1,355,955 (32.7\%) \\
			&ISOC   & 105,405 (28.0\%)  &  271,601 (72.0\%)   &   413,212 (70.4\%)  &  174,065 (29.6\%)  \\
			\midrule
			\multirow{2}{*}{\begin{sideways}Q\end{sideways}} 
			&INIST  & 1,331,980 (48.5\%)  & 1,416,885 (51.5\%)  & 3,779,837 (91.3\%)  & 360,591 (8.7\%) \\
			&ISOC   & 187,681 (49.8\%)  & 189,325 (50.2\%)  & 544,690 (92.7\%)  & 42,587 (7.3\%)\\
			\bottomrule
		\end{tabular}
		\caption{Number of IT\emph{lan}L translated with the multilingual MeSH (M) and the full QuadLexicon (Q) for English, German, French and Spanish.}
		\label{tab:tradsITL}
	\end{table}

	
	
	%
	% ---- Bibliography ----
	%
	\addcontentsline{toc}{section}{References}
	\bibliographystyle{plain}
	\bibliography{genericMT}
	
	
\end{document}
