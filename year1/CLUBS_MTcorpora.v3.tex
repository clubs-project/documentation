\documentclass[a4paper,11pt]{article}

\usepackage[utf8]{inputenc}
\usepackage{booktabs, array, pdflscape}
\usepackage{geometry}
\usepackage{graphics,subfigure,graphicx}
\usepackage{color}
\usepackage{url}
\usepackage{enumerate}

\setlength{\textheight}{24cm}  
\setlength{\textwidth}{15cm}
\setlength\oddsidemargin{0cm}
\setlength\evensidemargin{0cm}
\setlength\voffset{-1cm}

\renewcommand{\textfraction}{0.01}
\renewcommand{\floatpagefraction}{0.75}
\renewcommand{\topfraction}{0.8}
\renewcommand{\bottomfraction}{0.8}

\newcommand{\red}[1]{\textcolor{red}{#1}}	
\newcommand{\mc}[3]{\multicolumn{#1}{#2}{#3}}

\newcommand{\Ni}{({\em i\,})~}
\newcommand{\Nii}{({\em ii\,})~}
\newcommand{\Niii}{({\em iii\,})~}

%opening
\title{
\includegraphics[width=3cm]{./img/200px-SuitClubs.png} \\
\Huge M1.2 -- Corpora for the \\ Machine Translation Engines \\ 
}
\author{\vspace*{1cm}\\ \LARGE Cristina Espa\~na-Bonet$^1$, Juliane Stiller$^2$ and Sophie Henning$^1$ \medskip \\ \Large $^1$Universit\"at des Saarlandes, $^2$Humboldt-Universit\"at zu Berlin}
\date{\vspace*{2cm} -- v4.0 --\\July 2018}

%\author{\vspace*{1cm}\\ \LARGE Cristina Espa\~na-Bonet \medskip \\ \Large Universit\"at des Saarlandes}
%\date{\vspace*{2cm} -- v1.0 --\\December 2016}


\begin{document}

\clearpage\maketitle
\thispagestyle{empty}

\vspace*{5cm}
\begin{abstract}
This document describes the corpora used for training and evaluating the baseline MT engines used within the CLUBS project. We include as an appendix the modifications done for the final systems.
\end{abstract}

\newpage
\tableofcontents
\clearpage

% guarrada, no va el \cleardoublepage
% \clearpage\mbox{}\clearpage

%\newpage
\section{Introduction}
\label{s:intro}

The CLuBS project uses crosslingual information retrieval (CLIR) techniques to search documents in a psychology database, PubPsych\footnote{\url{https://www.pubpsych.eu}}. These crosslingual techniques need machine translation (MT) engines in order to translate the documents. In the project, we will develop several data-based machine translation systems and study their performance as translation engines, but also their effect within the retrieval system. This document describes the first selection of corpora used for training and evaluating the systems.

Since data-based MT relies on the existence of large amount of parallel corpora, we have gathered parallel and monolingual corpora not only on psychology, but also on several other domains, some rather close in topic, such as medicine, others, such as politics, farther away. Out-of-domain corpora are easier and cheaper to get and help to improve the quality of the translators, especially when data is scarce. In our case, which is translation between De--En--Es--Fr in the domain of psychology, we can find huge amounts of data both in-domain and out-of-domain for some of the language pairs (e.g. En--De). For some other pairs only out-of domain data is available (e.g. En--Fr). 
% Finally, some language pairs do not have parallel data at all (De--Es).

In the following sections, we describe the extraction of the monolingual and parallel corpus from the PubPsych database (Section~\ref{s:pub}), and the monolingual and parallel out-of-domain corpus gathered from several sources (Section~\ref{s:gen}). 
% Section~\ref{s:lex} presents a multilingual in-domain lexicon extracted automatically from Wikipedia and some standard thesauri used for psychology.
Section~\ref{s:evaluation} presents the extraction of the evaluation corpus for the intrinsic evaluation of the machine translation. Finally, we sketch the next steps with regards to the corpus acquisition in Section~\ref{s:conclusions}.

\section{PubPsych Corpora}
\label{s:pub}

At the time of writing this report\footnote{1st December, 2016.}, the PubPsych database consists of 958.726 records. This data has been exported from the database in XML format. From all the metadata available, we use titles and abstracts for developing the translation engines. Not all the records have a title and/or an abstract and, in case they do, they are not available for all the languages in parallel. Tables~\ref{tab:recordsMono} and \ref{tab:recordsMulti} show the global figures of the database. English is the most populated language followed by German. Spanish and French have a similar amount of documents which is an order of magnitude smaller than English. However, although almost all the documents in French have the title also in English or Spanish, the abstract is seldom parallel. In the case of Spanish, half of the articles have the abstract both in Spanish and English. That makes French a language with almost no resources to train an in-domain MT system. For English--German and English--Spanish we can build in-domain parallel corpora, for English--French and pairs not involving English, there is not enough data.


\subsection{Parallel Corpora Extraction}
\label{ss:pubPar}

We select the records with titles or abstracts in more than one language to extract two parallel corpus for each language pair with a significant representation: one containing titles and one with sentences of the abstracts. In both cases, some processing of the data is needed. The code used for this processing is available in the project's GitHub repository\footnote{\url{https://github.com/clubs-project/corpora-extraction-scripts}}.

\paragraph{Extraction of titles and subtitles.}
Records sometimes not only have a title but also a subtitle in a separate field. For building the parallel corpus one has to align these fields between the languages L1 and L2, because the title information in one language may be split in separate title and subtitle in another language. In general, one can consider four different cases:
\begin{enumerate}[(i)]
\itemsep0em 
 \item Only the title is available for L1 and L2
 \item L1 has title and subtitle, L2 only has title
 \item L1 only has title, L2 has title and subtitle
 \item Both L1 and L2 have title and subtitle
\end{enumerate}

In order to extract the appropriate pairs, we implement a heuristic based on the expected ratio of the length of sentences (length factor) between the two languages. We assume that the length factor follows a normal distribution and estimate its mean and standard deviation in out-of-domain parallel corpora\footnote{See corpora in Section~\ref{ss:genWeb}.}. Then, for a given parallel record, we generate all possible combinations between the available titles and subtitles and extract the pair with the closest length factor to the expected mean of the language pair.

We notice that all the cases where the title information is split into title and subtitle in PubPsych are from the Psyndex source database. The division can either be in the title of the English side or in the German one. Besides, there are no records where both L1 and L2 have title and subtitle.


\begin{table}[t]
 \centering
 \footnotesize
    \begin{tabular}{lllll}
    \toprule
    & \mc{1}{c}{De} & \mc{1}{c}{En} & \mc{1}{c}{Es} & \mc{1}{c}{Fr} \\
    \midrule
    Titles    & 324.005 & 895.982 & 53.065 & 47.707 \\
    Abstracts & 250.263 & 513.000 & 34.815 & 33.206 \\
    \bottomrule
    \end{tabular}
 \caption{Number of records with title or abstract for the four main languages in the PubPsych database.}
 \label{tab:recordsMono}
\end{table}

\begin{table}[t]
 \centering
 \footnotesize
  \begin{tabular}{lrrrrrrrr}
    \toprule
  & \mc{1}{c}{En--De} & \mc{1}{c}{En--Es} & \mc{1}{c}{En--Fr} & \mc{1}{c}{De--Es} & \mc{1}{c}{De--Fr} & \mc{1}{c}{Es--Fr}
  & \mc{1}{c}{En--Es--Fr} & \mc{1}{c}{De--En--Fr}\\
    \midrule
  Titles    & 307.37 & 25.680 & 45.324 & 7 & 50 &   2  &   2 & 6\\
  Abstracts & 47.218 & 16.934 &    189 & 0 &  0 & 105  & 105 & 0 \\
    \bottomrule
  \end{tabular}
 \caption{Multilinguality in the PubPsych database. Number of records with parallel titles or abstracts. There is no record with either the title or the abstract in all four languages.}
 \label{tab:recordsMulti}
\end{table}



\paragraph{Extraction of parallel sentences in abstracts.}

Parallel abstracts have to be aligned at sentence level in order to build the MT corpus. To do this, sentences are first split with an in-house sentence splitter and then passed to an aligner based on lengths and positions within the document. We use an available Python implementation of the Gale-Church algorithm%
\footnote{\url{https://github.com/alvations/NTU-MC/blob/master/ntumc/toolkit/gale_church.py}} \cite{galeChurch:1993} for this. After the alignment, only sentences with more than 5 words are used for the parallel corpus of abstracts; the restriction does not apply to titles. In both cases, sentences completely uppercased are converted into lowercase strings.



\subsection{Monolingual Corpora Extraction}
\label{ss:pubMono}

For building the monolingual corpora, we consider all the records in the database with either a title and/or an abstract and extract the text in English, French, German and Spanish. For abstracts, text snippets are split into sentences. For titles, both the title and the subtitle fields are extracted. We do not need to deal with alignment issues in this case and all the sentences can be gathered.



\subsection{Training, Development and Test Sets}
\label{ss:pubSets}

% The full corpus is divided into three parts: training, development and test. 

From the full corpus, we first put aside 800 records that will be translated manually to build a high-quality test set. These records are selected from the subset of documents with an English abstract only (i.e. without any translation) and then sampled randomly according to the weight of its data source (e.g. PSYNDEX, ISOC-Psicología, etc.). The remaining MT corpus consists of 957.926 documents.

We create a partition with 1500 documents on the MT corpus to build the development and test sets. For this partition, we give preference to
% we gather 1500 documents per language (English, French, German and Spanish) preferring 
records with multilingual abstracts and again sample the MT corpus randomly according to the weight of its data source. The division between development and test is done differently for every language pair with the purpose of having a similar amount of parallel sentences. We use 1500 parallel sentences from the abstracts for development and the rest for testing. All the titles are used to build a test set and no development set is considered as a first approximation. In the case of French, we can only build one set due to the small amount of parallel data. The remaining 951.926 documents with title and/or abstract are used for training.

Table~\ref{tab:setsParPubPshyc} shows the sizes for the in-domain parallel corpora obtained for each language pair after the division in training, development and test. Table~\ref{tab:setsMonoPubPshyc} shows the sizes of the in-domain monolingual corpora for the four languages involved; in this case, all data is used for training.


\begin{table}[t]
\footnotesize
\flushleft{\bf Abstracts\\ \smallskip}
\begin{tabular}{lrrrrrrrrr}
\toprule
   & \mc{3}{c}{En--De} & \mc{3}{c}{En--Es} & \mc{3}{c}{En--Fr}\\
   \cmidrule(lr){2-4}   \cmidrule(lr){5-7}   \cmidrule(lr){8-10}
   & \mc{1}{c}{Sent.} & \mc{1}{c}{EnTok.} & \mc{1}{c}{DeTok.} & \mc{1}{c}{Sent.} & \mc{1}{c}{EnTok.} & \mc{1}{c}{EsTok.} & \mc{1}{c}{Sent.} & \mc{1}{c}{EnTok.} & \mc{1}{c}{FrTok.}\\
\midrule
Train & 241.749 & 6.584.364 & 6.135.612 & 88.848 & 2.640.441 & 2.909.559 & 0 & 0 & 0\\
Dev.  &   1.500 &   39.968 & 37.557 & 1.500 & 45.611 & 50.831 & 0 & 0 & 0\\
Test  &   2.162 &   60.219 & 55.610 & 2.486 & 74.382 & 81.575 & 823 & 25.884 & 29.226 \\
\bottomrule
\end{tabular}

\flushleft{\bf Titles\\ \smallskip}
\begin{tabular}{lrrrrrrrrr}
\toprule
   & \mc{3}{c}{En--De} & \mc{3}{c}{En--Es} & \mc{3}{c}{En--Fr}\\
   \cmidrule(lr){2-4}   \cmidrule(lr){5-7}   \cmidrule(lr){8-10}
   & \mc{1}{c}{Sent.} & \mc{1}{c}{EnTok.} & \mc{1}{c}{DeTok.} & \mc{1}{c}{Sent.} & \mc{1}{c}{EnTok.} & \mc{1}{c}{EsTok.} & \mc{1}{c}{Sent.} & \mc{1}{c}{EnTok.} & \mc{1}{c}{FrTok.}\\
\midrule
Train & 306.640 & 3.480.727 & 3.059.048 & 25.105 & 293.164 & 340.203 & 45.137 & 463.610 & 567.618\\
Dev.  &      0 &       0 &       0 &     0 &      0 &      0 &     0 &      0 & 0\\
Test  &    737 &    9.691 &    8.202 &   575 &   6.935 &   8.002 &   187 &   2.589 & 3.012 \\
\bottomrule
\end{tabular}

 \caption{Size of the parallel corpora obtained from the PubPsych database and used for training, development and testing the MT systems.}
 \label{tab:setsParPubPshyc}
\end{table}



\begin{table}[t]
\footnotesize
\centering
\bigskip
\flushleft{\bf Abstracts \& Titles\\ \smallskip}
\begin{tabular}{l rrrrrrrr}
\toprule
 & \mc{2}{c}{De} & \mc{2}{c}{En} & \mc{2}{c}{Es} & \mc{2}{c}{Fr}\\
   \cmidrule(lr){2-3}   \cmidrule(lr){4-5}  \cmidrule(lr){6-7}   \cmidrule(lr){8-9}
 & \mc{1}{c}{Sent.} & \mc{1}{c}{Tok.} & \mc{1}{c}{Sent.} & \mc{1}{c}{Tok.} & \mc{1}{c}{Sent.} & \mc{1}{c}{Tok.} & \mc{1}{c}{Sent.} & \mc{1}{c}{Tok.}\\
\midrule
Abst.    & 1.618.845 & 36.093.889 & 3.533.855 & 92.211.688 & 168.423 & 5.393.989 & 150.537 & 4.571.979\\
Titles    &  396.152 & 3.343.386 & 896.876 & 10.741.170 & 52.399 & 631.757 & 47.529 & 595.249 \\
\bottomrule
\end{tabular}
 \caption{Size of the monolingual corpora obtained from the PubPsych database and used for training the MT systems.}
 \label{tab:setsMonoPubPshyc}
\end{table}


\subsection{Pre-Processing}
\label{ss:pubPreprp}

In order to use the previous corpora in natural language applications, text must be pre-processed. We follow a three-step pre-processing standard in MT here:

\begin{enumerate}
\itemsep0em 
 \item Text normalisation
 \item Text tokenisation
 \item Truecasing
\end{enumerate}

For text normalisation and text tokenisation we use scripts from the Moses toolkit~\cite{moses:2007}. The normaliser, however, has been adapted to deal with some peculiarities of the PubPsych corpus. For truecasing, we have trained statistical models using Wikipedia and Europarl~V7 monolingual corpora (Section~\ref{s:gen}).




\section{Out-of-Domain Corpora}
\label{s:gen}

The data described in the previous section is not enough to train high-quality translators, but it is important to enrich larger corpora with the language of psychology itself. In the following, we describe several open parallel corpus that will be used within the project to train general and specialised translators.
% The size of these corpus can be seen in Tables XXX.

\subsection{Medicine, Biological and Health domains}

First, we describe two corpora that can be considered to be rather close in domain to psychology: the EMEA~\cite{tiedemann:2009} and the Scielo corpus.

\paragraph{EMEA Corpus.}
This is a parallel corpus made out of PDF documents from the European MEdicines Agency\footnote{\url{http://www.emea.europa.eu}}. Files were automatically converted from PDF to plain text, sentence-aligned and made publicly available within the Opus Corpus~\cite{tiedemann:2009}. The corpus is available in 22 languages including English, French, German and Spanish. At the moment, we only consider the language pairs involving English to be compatible with the other corpora, but this corpus could be used later in the project for all the language pairs.

\paragraph{Scielo Corpus.}
This is a parallel corpus made out of documents retrieved from the SCIentific Electronic Library Online\footnote{\url{http://www.scielo.org}}. The documents belong to the biological and health domains and can be composed of either a title, an abstract or both of them. The corpus was prepared by the organisers of the Biomedical Translation Task in the First Conference on Machine Translation\footnote{\url{http://www.statmt.org/wmt16/biomedical-translation-task.html}} (WMT16), who aligned the documents at sentence level with the GMA tool\footnote{\url{http://nlp.cs.nyu.edu/GMA}}. In this case, the corpus is available for the English--Spanish and English--French language pairs.



\begin{landscape}
\begin{table}[t]
% \small
\begin{tabular}{l rrr rrr rrr}
\toprule
    & \mc{3}{c}{En--De} & \mc{3}{c}{En--Es} & \mc{3}{c}{En--Fr}\\
    \cmidrule(lr){2-4}   \cmidrule(lr){5-7}   \cmidrule(lr){8-10} 
    & \mc{1}{c}{Sent.} & \mc{1}{c}{EnTok.} & \mc{1}{c}{DeTok.} 
    & \mc{1}{c}{Sent.} & \mc{1}{c}{EnTok.} & \mc{1}{c}{EsTok.} 
    & \mc{1}{c}{Sent.} & \mc{1}{c}{EnTok.} & \mc{1}{c}{FrTok.}\\
\midrule
UN           &   162.981 &   6.098.083 &  5.617.876 & 11.196.913 & 320.064.682 & 366.072.923 & 12.886.831 & 361.877.676 & 421.687.471\\
EP           & 1.920.209 &  53.091.548 & 50.548.739 &  1.965.734 &  54.505.707 &  57.047.216 &  2.007.723 &  55.730.752 &  61.888.789\\
ComCrawl     & 2.399.123 &  58.864.439 & 54.570.779 &  1.845.286 &  46.855.705 &  49.557.537 &  3.244.152 &  81.084.856 &  91.281.890\\
\it{subTOTAL}       & \it{4.482.313} &  \it{118.054.070} & \it{110.737.394} &\it{15.007.933} &  \it{421.426.094} & \it{472.677.676} &  \it{18.138.706} & \it{498.693.284}  & \it{574.858.150} \\
\midrule
EMEA         & 1.108.752 &  14.477.119 & 13.197.725 &  1.098.333 &  14.334.648 &  15.975.506 &  1.092.568 &  14.317.365 & 17.046.979\\
ScieloBio    & -- & -- & -- &  117.862 & 3.252.183 & 3.382.511 & -- & -- & --\\
ScieloHealth & -- & -- & -- &  558.714 & 14.382.853 & 15.031.533 & 9.129 & 244.486 & 308.055\\
\it{subTOTAL}  & \it{1.108.752} &  \it{14.477.119} & \it{13.197.725} & \it{1.774.909} & \it{31.969.684} & \it{34.389.550} & \it{1.101.697}  & \it{14.561.851}  & \it{17.355.034}  \\
\midrule
PubPsych     &  241.749 & 6.584.364 & 6.135.612 & 88.848 & 2.640.441 & 2.909.559 & -- & -- & --\\
\midrule
\it{TOTAL}  & \it{5.832.814} & \it{139.115.553} & \it{130.070.731} & \it{16.871.690}  & \it{456.036.219}  & \it{509.976.785} & \it{19.240.403} & \it{513.255.135} & \it{592.213.184}  \\
\bottomrule
\end{tabular}
 \caption{Size of the parallel corpora obtained from the different sources described in Section~\ref{ss:pubPar}: United Nations (UN), Europarl V7 (EP), Common Crawl (ComCrawl), EMEA and Scielo. Figures of the PubPsych corpus are shown for comparison.}
 \label{tab:setsParGen}
\end{table}


\begin{table}[t]
% \footnotesize
\centering
\begin{tabular}{l rrrrrrrr}
    \toprule
    & \mc{2}{c}{De} & \mc{2}{c}{En} & \mc{2}{c}{Es} & \mc{2}{c}{Fr}\\
      \cmidrule(lr){2-3}   \cmidrule(lr){4-5}  \cmidrule(lr){6-7}   \cmidrule(lr){8-9}
    & \mc{1}{c}{Sent.} & \mc{1}{c}{Tok.} & \mc{1}{c}{Sent.} & \mc{1}{c}{Tok.} & \mc{1}{c}{Sent.} & \mc{1}{c}{Tok.} & \mc{1}{c}{Sent.} & \mc{1}{c}{Tok.}\\
    \midrule
    Wikipedia   &39.036.439 &675.868.710 & 92.284.575 & 1.920.645.814 &20.085.435&465.828.442 & 26.603.296  & 553.201.962\\
    General     & 4.482.313 &110.737.394 & 18.138.706 &   498.693.284 &15.007.933&472.677.676 & 19.240.403 & 592.213.184 \\
    Medicine    & 1.108.752 & 13.197.725 & 1.774.909  &    31.969.684 & 1.774.909& 34.389.550 &  1.101.697 &  17.355.034 \\
    PubPsych    & 1.618.845 & 36.093.889 & 3.533.855  &    92.211.688 &  168.423 &  5.393.989 &    150.537 &   4.571.979\\
    \bottomrule
\end{tabular}
\caption{Size of the monolingual corpora obtained from the different sources described in Section~\ref{ss:pubPar}. \emph{General} includes the UN, EP and ComCrawl corpora and \emph{Medicine} includes the EMEA and Scielo ones.}
\label{tab:setsMonoGen}
\end{table}


\end{landscape}


\subsection{General Corpora: Web and Politics}
\label{ss:genWeb}

Second, we use corpora from very distant domains to gather general phrases. These corpora include texts on politics (Europarl Corpus~\cite{europarl:2005} and United Nations Corpus~\cite{multiun:2012}) and crawls from the web (Common Crawl corpus and Wikipedia articles). Similarly to EMEA, we only consider the language pairs involving English in the Europarl (EP) and United Nations (UN) corpora, but they could be used later in the project for all the language pairs. The Common Crawl parallel corpus is made publicly available by the annual Shared Task on Machine Translation (WMT), currently Conference on Machine Translation. The Wikipedia corpus is a monolingual in-house corpus with dumps downloaded from Wikipedia\footnote{\url{https://dumps.wikimedia.org}} in January 2015 and pre-processed with JWPL\footnote{\url{https://code.google.com/p/jwpl}}~\cite{zeschEtal:2008}.

\begin{table}[t]
\footnotesize
\centering
\begin{tabular}{l rrr rrr rrr}
\toprule
    & \mc{3}{c}{En--De} & \mc{3}{c}{En--Es} & \mc{3}{c}{En--Fr}\\
    \cmidrule(lr){2-4}   \cmidrule(lr){5-7}   \cmidrule(lr){8-10} 
    & \mc{1}{c}{Sent.} & \mc{1}{c}{EnTok.} & \mc{1}{c}{DeTok.} 
    & \mc{1}{c}{Sent.} & \mc{1}{c}{EnTok.} & \mc{1}{c}{EsTok.} 
    & \mc{1}{c}{Sent.} & \mc{1}{c}{EnTok.} & \mc{1}{c}{FrTok.}\\
\midrule
news-test2012  & 3.003 & 72.988 & 72.603 & 3.003 & 72.988 & 78.887 & 3.003 & 72.988 & 81.797 \\
news-test2013  & 3.000 & 64.809 & 63.411 & 3.000 & 64.809 & 70.540 & 3.000 & 64.809 & 73.658 \\
EMEAdev        & 2.000 & 38.658 & 37.945 & 2.000 & 36.676 & 39.959 & 2.000 & 34.554 & 41.026 \\
EMEAtest       & 2.000 & 36.864 & 35.773 & 2.000 & 34.359 & 38.615 & 2.000 & 33.316 & 39.674 \\
PubPsychDev    & 1.500 & 39.968 & 37.557 & 1.500 & 45.611 & 50.831 & -- & -- & --\\
PubPsychTest   & 2.162 & 60.219 & 55.610 & 2.486 & 74.382 & 81.575 & 823 & 25.884 & 29.226 \\
\bottomrule
\end{tabular}
 \caption{Size of the out-of-domain test sets to be used in the project. Figures of the automatic PubPsych test sets are shown for comparison.}
 \label{tab:testsParGen}
\end{table}



Tables~\ref{tab:setsParGen} and \ref{tab:setsMonoGen} show a summary of the amount of data available for every language pair. English--French is the pair with most parallel data, more than 19 million parallel sentences, but all of them belong to out-of-domain data. EMEA and Scielo are the closest sources in this case to obtain in-domain vocabulary together with the addition of the thesauri used in psychology (see Deliverable 3.1). On the other side lies the English--German language pair. Here we have less than 6 million parallel sentences but almost 250 thousand belonging to PubPsych. Finally, there are almost 17 million parallel sentences for the English--Spanish pair, with less than 100 thousand belonging to psychology.

%      117862     3252183 en-es/scielo.en-es.bio.tc.en
%      117862     3382511 en-es/scielo.en-es.bio.tc.es
%      558714    14382853 en-es/scielo.en-es.health.tc.en
%      558714    15031533 en-es/scielo.en-es.health.tc.es
%        9129      244486 en-fr/scielo.en-fr.health.tc.en
%        9129      308055 en-fr/scielo.en-fr.health.tc.fr

%     1108752    13197725 en-de/EMEA.de-en.tc.de
%     1108752    14477119 en-de/EMEA.de-en.tc.en
%     1098333    14334648 en-es/EMEA.en-es.tc.en
%     1098333    15975506 en-es/EMEA.en-es.tc.es
%     1092568    14317365 en-fr/EMEA.en-fr.tc.en
%     1092568    17046979 en-fr/EMEA.en-fr.tc.fr

%     2399123    54570779 en-de/commoncrawl.de-en.tc.de
%     2399123    58864439 en-de/commoncrawl.de-en.tc.en
%     1845286    46855705 en-es/commoncrawl.es-en.tc.en
%     1845286    49557537 en-es/commoncrawl.es-en.tc.es
%     3244152    81084856 en-fr/commoncrawl.fr-en.tc.en
%     3244152    91281890 en-fr/commoncrawl.fr-en.tc.fr


%     1920209    50548739 en-de/europarl-v7.de-en.tc.de
%     1920209    53091548 en-de/europarl-v7.de-en.tc.en
%     1965734    54505707 en-es/europarl-v7.es-en.tc.en
%     1965734    57047216 en-es/europarl-v7.es-en.tc.es
%     2007723    55730752 en-fr/europarl-v7.fr-en.tc.en
%     2007723    61888789 en-fr/europarl-v7.fr-en.tc.fr

%      162981     5617876 en-de/MultiUN.de-en.tc.de
%      162981     6098083 en-de/MultiUN.de-en.tc.en
%    11196913   320064682 en-es/undoc.2000.es-en.tc.en
%    11196913   366072923 en-es/undoc.2000.es-en.tc.es
%    12886831   361877676 en-fr/undoc.2000.fr-en.tc.en
%    12886831   421687471 en-fr/undoc.2000.fr-en.tc.fr
%    
%    18968763   483197115 en-fr/generalBio.fr-en.tc.clean.en
%    18968763   556646759 en-fr/generalBio.fr-en.tc.clean.fr
%    19240403   513255135 en-fr/generalBio.fr-en.tc.en
%    19240403   592213184 en-fr/generalBio.fr-en.tc.fr

For the second version of the corpus, we include data for the language pairs not involving English (See Apendix~\ref{s:additional}) which is only available for the out-of-domain setting. We use Europarl~\cite{europarl:2005}, United Nations Corpus~\cite{multiun:2012}, JR-ACQUIS~\cite{tiedemann:2012} and News Commentary~\cite{tiedemann:2012} for that purpose. For somehow alleviating the scarcity of data in German, we also include the MODEL Rapid corpus~\cite{rozisSkadins:2017} for all the language pairs involving German. Finally, we include the first 2 million parallel sentences from the EUbookshop~\cite{skadinsEtal:2014} for German--Spanish and German--French. The latter two sources are supposed to be more noisy than the previous ones and for this reason we only include them for the pairs with less data.


\subsection{Development and Test Sets}
\label{ss:genTests}

For development and testing we consider four sets, two corresponding to what we have called general domain and two more from the medical domain. For the general domain we use \emph{news-test2012} and \emph{news-test2013}. These test sets are distributed by the organisation of the WMT workshops. The years selected for our experiments are the last ones to include English, French, German and Spanish simultaneously, so that the sets are aligned among the four languages. For the medical domain we use a subset of the EMEA corpus. In this case, the sets are only aligned inside a language pair.

Table~\ref{tab:testsParGen} shows the figures for these dev/test sets and the comparison with the test sets gathered automatically from the PubPsych data. 
The upcoming human translated test set will allow the evaluation of all the language pairs, also for in-domain translations.

% The upcoming test set manually generated with human translations of 800 PubPsych records will allow the evaluation of all the language pairs, also those not involving English.


\subsection{Pre-Processing}
\label{ss:genPreprp}

All the corpora introduced in this section has been pre-processed with the same pipeline and tools outlined in Section~\ref{ss:pubPreprp}.

% 
% \section{Multilingual Lexicons and Thesauri}
% \label{s:lex}
% 
% We have extracted multilingual in-domain phrases from Wikipedia with the WikiTailor tool\footnote{\url{https://github.com/cristinae/WikiTailor}} \cite{barronEtAl:2015}. Lexicons have been built from parallel titles of articles in the psychology and health domains for English, German, French and Spanish using WikiTailor models WT0.5-100 or WT0.5-500 depending on the language.
% 
% Using the intersection of in-domain articles in the four languages we obtain a high precision/low recall multilingual lexicon with 497 entries. With the union of in-domain articles we gather a low precision/high recall multilingual lexicon with 81.369 entries. The lexicon contains both single words and phrases:
% 
% \bigskip
%  \begin{small}
% % \begin{tabular}[h]{rl rl rl rl}
% %   ID En    & Title En       & ID Es   & Title Es        &  ID Fr    & Title Fr            &  ID De   & Title De \\
% %   1053998  & Perception     &176025   & Percepci\'on    & 360640    &Perception           & 387646   & Wahrnehmung \\
% %   10269587 & Echoic\_memory & 4636204 & Memoria\_ecoica & 3858294   & M\'emoire\_auditive & 4334978  & Echoisches\_Ged\"achtnis \\
% % \end{tabular}
%   \begin{tabular}[h]{llll}
%     \toprule
%     En     &Es          & Fr             & De \\
%     \midrule
%     Perception     & Percepci\'on      &Perception            & Wahrnehmung \\
%     Echoic\_memory & Memoria\_ecoica   & M\'emoire\_auditive  & Echoisches\_Ged\"achtnis \\
%     ... & ... & ... & ... \\
%     \bottomrule
%  \end{tabular}
% \end{small}
% \bigskip
% \bigskip
%  
% On the other hand, several thesauri are used to tag content in the PubPsych database. The specific thesaurus (APA Thesaurus of Psychological Index Terms (possibly in translation), MeSH Medical Subject Headings, TermSciences, etc.) depends on the original source database of the document. At least the APA and MeSH thesauri are multilingual and can be used, within the translation task, to enrich the in-domain vocabulary.
% This paper describes the UdS-DFKI participation to the multilingual task of the IWSLT Evaluation 2017. Our approach is based on factored multilingual neural translation systems following the small data and zero-shot training conditions. Our systems are designed to fully exploit multilinguality by including factors that increase the number of common elements among languages such as phonetic coarse encodings and synsets, besides shallow part-of-speech tags, stems and lemmas. Document level information is also considered by including the topic of every document. This approach improves a baseline without any additional factor for all the language pairs and even allows beyond-zero-shot translation. That is, the translation from unseen languages is possible thanks to the common elements &mdash;especially synsets in our models&mdash; among languages.


\section{Corpora for Evaluation}
\label{s:evaluation}
To intrinsically evaluate the machine translation with scores like BLEU ~\cite{papineni_bleu:_2002}, a corpus of 800 (metadata)records from PubPsych aligned in the four languages English, Spanish, German and French is provided. For that, 800 records with only one English abstract were extracted from the full corpus. They will be translated into the respective other three languages in the near future.

The corpus for evaluation was extracted from the full data file provided on Nov. 19, 2016, with a total of 958.726 records. From these, we extracted the ones with only one English abstract and no parallel translations resulting in a corpus of 448.764 records. From these records, we sampled 800 records randomly based on the weighting of the data source. Table \ref{tab:eval800} lists the properties per source database and the percentage and corresponding number of records in the evaluation corpus.

\begin{small}
\begin{table}
\centering
\begin{tabular}{llll}
\toprule
Data source & \# of records & \# of recs with 1 EN abstract & \# for sample \\
\midrule
ERIC 		& 107.532 		& 104.648  & 187 (23\%) \\
PSYNDEX    & 331.469        & 56.202 & 100 (13\% )   \\
PASCAL      & 206.670        & 132.439 & 236 (30\%)           \\
NARCIS      & 29.847         & 13.734  & 24 (3\% )          \\
ISOC        & 50.275         & 705  & 1 (0\%)         \\
PSYCHOPEN   & 1.059          & 976    & 2  (0\%)           \\
PSYCHDATA   & 53            & 1  & 0  (0\%)           \\
NORART      & 11.443         & 0  & 0 (0\%)             \\
\midrule
TOTAL  &  958.726        & 448.732   & 800 (100\%)\\
\bottomrule
\end{tabular}
\caption{Number of extracted documents for the evaluation corpus based on source weighting.}
\label{tab:eval800}
\end{table}
\end{small}

Table \ref{tab:publang} shows stats on the extracted corpus grouped by the publication language. 

\begin{small}
\begin{table}[]
\centering
\begin{tabular}{lllllllll}
\toprule
pub\_lang  & records  & subtitle\_en & title\_de & title\_en & title\_es & title\_fr &  \\
\midrule
chi               & 1            & 0            & 0         & 1         & 0         & 0         &  \\
dut               & 1            & 0            & 0         & 1         & 0         & 0         &  \\
en               & 24           & 0            & 0         & 24        & 0         & 0         &  \\
eng               & 711          & 3            & 87        & 711       & 1         & 0         &  \\
fre               & 7            & 0            & 0         & 6         & 0         & 7         &  \\
ger               & 11           & 0            & 11        & 11        & 0         & 0         &  \\
ita               & 1            & 0            & 0         & 1         & 0         & 0         &  \\
jpn              & 2            & 0            & 0         & 2         & 0         & 0         &  \\
por           & 1            & 0            & 0         & 1         & 0         & 0         &  \\
spa          & 4            & 0            & 0         & 4         & 4         & 0         &  \\
\bottomrule
\end{tabular}
\caption{Publication language of original documents in the evaluation corpus.}
\label{tab:publang}
\end{table}
\end{small}




\section{Conclusions}
\label{s:conclusions}

We have gathered a first collection of corpora to build the baseline MT systems for the project. The performance of the baseline systems will help us to detect new necessities on corpora in order to improve the systems. Possible weak points of the current selection include:

\begin{enumerate}[(i)]
\itemsep0em 
 \item SMT engines built using pivot methodologies do not perform well enough. In this case, we should consider German--French, German--Spanish, French--Spanish corpora when they are available
 \item MT engines show a very low quality when translating titles. In this case, we should train specific systems for titles and therefore prepare additional development/test sets
 \item We observe some untranslated words due to truecasing. In this case, a lowercased version of the corpora would be used.
\end{enumerate}

And already detected necessities:
\begin{enumerate}[(i)]
\itemsep0em 
 \item PubPsych human translated test sets fully aligned to test in-domain translation in all the language pairs
 \item Multilingual thesauri terms to add to current parallel corpora
\end{enumerate}


\appendix                                     
\section{Additional Corpora}
\label{s:additional}

For the last version of translator we have gathered additional corpora in German--English, German--French, German--Spanish and French--Spanish in order to improve the weak points of the previous corpus reported in the conclusions of the document. These data are only out-of-domain as PubPshyc does not have aligned documents for these pairs.

We have also improved the pre-processing of the data after some issues detected in the first corpus. The current pre-processing includes:

\begin{enumerate}
\itemsep0em 
 \item Removal of sentences where more tha $50\%$ of the text is in a non-latin alphabet (this removes equations and sentences in other languages such as Greek or Chinese)
 \item HTML entities cleaning (for instance, correcting double conversions such as aposaposquote; instead of \&aposquote;)
 \item Text normalisation
 \item Text tokenisation
 \item Truecasing
 \item Duplicate removal
\end{enumerate}

\noindent
The full pre-processing pipeline is available in the git repository of the project\footnote{\url{https://github.com/clubs-project/corpora-extraction-scripts/tree/master/cleaning}}.
Major differences come for the EMEA data, a corpus with lots of duplicate parallel sentences and equations. In this case, we finally get a clean subcorpus with only a quarter of the orginal sentences.


\subsection{Parallel Corpora Figures}


\begin{landscape}

\begin{table}[t]
% \small
\begin{tabular}{l rrr rrr rrr}
\toprule
    & \mc{3}{c}{En--De} & \mc{3}{c}{En--Es} & \mc{3}{c}{En--Fr}\\
    \cmidrule(lr){2-4}   \cmidrule(lr){5-7}   \cmidrule(lr){8-10} 
    & \mc{1}{c}{Sent.} & \mc{1}{c}{EnTok.} & \mc{1}{c}{DeTok.} 
    & \mc{1}{c}{Sent.} & \mc{1}{c}{EnTok.} & \mc{1}{c}{EsTok.} 
    & \mc{1}{c}{Sent.} & \mc{1}{c}{EnTok.} & \mc{1}{c}{FrTok.}\\
\midrule
UN             &     114.085 &   5.001.920 &  4.555.627 & 8.229.875 & 256.932.146 & 294.068.300 & 9.091.417 & 282.087.198 & 329.282.596\\
EP             &   1.862.433 &  52.674.125 & 50.126.351 & 1.910.098 &  54.065.460 &  56.618.936 & 1.956.314 &  55.312.925 &  60.886.010\\
ComCrawl       &   2.394.117 &  58.649.429 & 54.440.232 & 1.840.956 &  46.659.115 &  49.438.475 & 3.231.507 &  80.589.534 &  89.562.986\\
NC             &     221.211 &   5.502.983 &  5.608.068 &   236.969 &   6.019.121 &   6.881.068 &   207.894 &   5.237.637  &  6.240.595 \\
JRC            &     473.128 &  15.629.431 & 14.050.433 &  -- & -- & -- &  -- & -- & -- \\
Rapid          &   1.010.295 &  22.707.408 & 22.151.469 &  -- & -- & -- &  -- & -- & -- \\
\it{subTOTAL}  & \it{6.075.269} &  \it{160.165.296} & \it{150.932.180}&  \it{12.217.898} & \it{363.675.842}  & \it{407.006.779} &\it{14.487.132} &  \it{423.227.294} & \it{485.972.187}  \\
\midrule
EMEA         & 237.607 &  4.111.481 & 3.779.495 &  240.888 &  4.229.668 &  4.697.635 &  246.315 &  4.258.733 & 5.036.208\\
ScieloBio    & -- & -- & -- &  116.999 & 3.246.340 & 3.376.848 & -- & -- & --\\
ScieloHealth & -- & -- & -- &  532.287 & 14.016.862 & 14.649.695 & 8.990 & 242.750 & 304.494\\
\it{subTOTAL}  &  \it{237.607} &  \it{4.111.481} & \it{3.779.495} & \it{890.174} & \it{2.149.2870} & \it{22.724.178} & \it{255.305}  & \it{4.501.483}  & \it{5.340.702}  \\
\midrule
PubPsych     &  542.690 & 10.023.347 & 9.161.397 & 112.148 & 2.907.160 & 3.222.127 & 44.505 & 457.745 & 560.716\\
\midrule
\it{TOTAL}  & \it{6.753.879} &  \it{148.073.799} & \it{137.456.811} & \it{13.220.220}  & \it{388.075.872}  & \it{432.953.084} & \it{14.786.942} & \it{428.186.522} & \it{491.873.605}  \\
\bottomrule
\bigskip\\
\toprule
    & \mc{3}{c}{De--Es} & \mc{3}{c}{De--Fr} & \mc{3}{c}{Es--Fr}\\
    \cmidrule(lr){2-4}   \cmidrule(lr){5-7}   \cmidrule(lr){8-10} 
    & \mc{1}{c}{Sent.} & \mc{1}{c}{DeTok.} & \mc{1}{c}{EsTok.} 
    & \mc{1}{c}{Sent.} & \mc{1}{c}{DeTok.} & \mc{1}{c}{FrTok.} 
    & \mc{1}{c}{Sent.} & \mc{1}{c}{EsTok.} & \mc{1}{c}{FrTok.}\\
\midrule
UN             &   120.334 &  4.773.923 &  6.115.036 &  120.529 &  4.736.002 &  6.111.522 & 8.488.449 & 307.870.137 & 314.689.536\\
EP             & 1.842.324 & 49.824.584 & 54.749.655 &1.901.615 & 51.051.092 & 58.973.416 & 1.937.659 &  57.569.936 &  60.481.886\\
JRC            &   963.545 & 28.133.439 & 33.757.484 &  951.566 & 27.759.441 & 33.481.936 &   962.918 &  33.533.403 &  33.674.993\\
NC             &   209.534 &  5.453.601 &  6.119.785 &  185.039 &  4.763.166 &  5.581.034 &   194.943 &  5.763.484 &  6.008.991\\
Rapid          &   526.605 & 11.499.817 & 13.484.752 &  975.509 & 21.045.994 & 25.922.810 &  -- & -- & -- \\
\it{subTOTAL}  & \it{3.662.342} &  \it{99.685.364} & \it{114.226.712}&  \it{4.134.258} & \it{109.355.695}  & \it{130.070.718} &\it{11.583.969} &  \it{404.736.960} & \it{414.855.406}  \\
\midrule
EMEA         & 258.428 &  4.138.271 & 4.988.972 &  265.202 &  4.171.805 &  5.354.591 &  265.230 &  5.117.314 & 5.472.197\\
\midrule
\it{TOTAL}  & \it{3.920.770} & \it{103.823.635} & \it{119.215.684} & \it{4.399.460}  & \it{113.527.500}  & \it{135.425.309} & \it{11.849.199} & \it{409.854.274} & \it{420.327.603}  \\
\bottomrule
\end{tabular}
 \caption{Size of the parallel corpora after the new cleaning pipeline obtained from the different sources described in Section~\ref{ss:pubPar}  with  (Multi-) United Nations (UN), Europarl V7 (EP), Common Crawl (ComCrawl), JRC-Acquis (JRC), MODEL Rapid (Rapid), EMEA and Scielo. Figures of the PubPsych corpus are shown for comparison.}
 \label{tab:setsParFinalEn}
\end{table}


\end{landscape}



%
% ---- Bibliography ----
%
\newpage
\addcontentsline{toc}{section}{References}
\bibliographystyle{plain}
\bibliography{genericMT}


\end{document}
